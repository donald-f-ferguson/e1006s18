
% Default to the notebook output style

    


% Inherit from the specified cell style.




    
\documentclass[11pt]{article}

    
    
    \usepackage[T1]{fontenc}
    % Nicer default font (+ math font) than Computer Modern for most use cases
    \usepackage{mathpazo}

    % Basic figure setup, for now with no caption control since it's done
    % automatically by Pandoc (which extracts ![](path) syntax from Markdown).
    \usepackage{graphicx}
    % We will generate all images so they have a width \maxwidth. This means
    % that they will get their normal width if they fit onto the page, but
    % are scaled down if they would overflow the margins.
    \makeatletter
    \def\maxwidth{\ifdim\Gin@nat@width>\linewidth\linewidth
    \else\Gin@nat@width\fi}
    \makeatother
    \let\Oldincludegraphics\includegraphics
    % Set max figure width to be 80% of text width, for now hardcoded.
    \renewcommand{\includegraphics}[1]{\Oldincludegraphics[width=.8\maxwidth]{#1}}
    % Ensure that by default, figures have no caption (until we provide a
    % proper Figure object with a Caption API and a way to capture that
    % in the conversion process - todo).
    \usepackage{caption}
    \DeclareCaptionLabelFormat{nolabel}{}
    \captionsetup{labelformat=nolabel}

    \usepackage{adjustbox} % Used to constrain images to a maximum size 
    \usepackage{xcolor} % Allow colors to be defined
    \usepackage{enumerate} % Needed for markdown enumerations to work
    \usepackage{geometry} % Used to adjust the document margins
    \usepackage{amsmath} % Equations
    \usepackage{amssymb} % Equations
    \usepackage{textcomp} % defines textquotesingle
    % Hack from http://tex.stackexchange.com/a/47451/13684:
    \AtBeginDocument{%
        \def\PYZsq{\textquotesingle}% Upright quotes in Pygmentized code
    }
    \usepackage{upquote} % Upright quotes for verbatim code
    \usepackage{eurosym} % defines \euro
    \usepackage[mathletters]{ucs} % Extended unicode (utf-8) support
    \usepackage[utf8x]{inputenc} % Allow utf-8 characters in the tex document
    \usepackage{fancyvrb} % verbatim replacement that allows latex
    \usepackage{grffile} % extends the file name processing of package graphics 
                         % to support a larger range 
    % The hyperref package gives us a pdf with properly built
    % internal navigation ('pdf bookmarks' for the table of contents,
    % internal cross-reference links, web links for URLs, etc.)
    \usepackage{hyperref}
    \usepackage{longtable} % longtable support required by pandoc >1.10
    \usepackage{booktabs}  % table support for pandoc > 1.12.2
    \usepackage[inline]{enumitem} % IRkernel/repr support (it uses the enumerate* environment)
    \usepackage[normalem]{ulem} % ulem is needed to support strikethroughs (\sout)
                                % normalem makes italics be italics, not underlines
    

    
    
    % Colors for the hyperref package
    \definecolor{urlcolor}{rgb}{0,.145,.698}
    \definecolor{linkcolor}{rgb}{.71,0.21,0.01}
    \definecolor{citecolor}{rgb}{.12,.54,.11}

    % ANSI colors
    \definecolor{ansi-black}{HTML}{3E424D}
    \definecolor{ansi-black-intense}{HTML}{282C36}
    \definecolor{ansi-red}{HTML}{E75C58}
    \definecolor{ansi-red-intense}{HTML}{B22B31}
    \definecolor{ansi-green}{HTML}{00A250}
    \definecolor{ansi-green-intense}{HTML}{007427}
    \definecolor{ansi-yellow}{HTML}{DDB62B}
    \definecolor{ansi-yellow-intense}{HTML}{B27D12}
    \definecolor{ansi-blue}{HTML}{208FFB}
    \definecolor{ansi-blue-intense}{HTML}{0065CA}
    \definecolor{ansi-magenta}{HTML}{D160C4}
    \definecolor{ansi-magenta-intense}{HTML}{A03196}
    \definecolor{ansi-cyan}{HTML}{60C6C8}
    \definecolor{ansi-cyan-intense}{HTML}{258F8F}
    \definecolor{ansi-white}{HTML}{C5C1B4}
    \definecolor{ansi-white-intense}{HTML}{A1A6B2}

    % commands and environments needed by pandoc snippets
    % extracted from the output of `pandoc -s`
    \providecommand{\tightlist}{%
      \setlength{\itemsep}{0pt}\setlength{\parskip}{0pt}}
    \DefineVerbatimEnvironment{Highlighting}{Verbatim}{commandchars=\\\{\}}
    % Add ',fontsize=\small' for more characters per line
    \newenvironment{Shaded}{}{}
    \newcommand{\KeywordTok}[1]{\textcolor[rgb]{0.00,0.44,0.13}{\textbf{{#1}}}}
    \newcommand{\DataTypeTok}[1]{\textcolor[rgb]{0.56,0.13,0.00}{{#1}}}
    \newcommand{\DecValTok}[1]{\textcolor[rgb]{0.25,0.63,0.44}{{#1}}}
    \newcommand{\BaseNTok}[1]{\textcolor[rgb]{0.25,0.63,0.44}{{#1}}}
    \newcommand{\FloatTok}[1]{\textcolor[rgb]{0.25,0.63,0.44}{{#1}}}
    \newcommand{\CharTok}[1]{\textcolor[rgb]{0.25,0.44,0.63}{{#1}}}
    \newcommand{\StringTok}[1]{\textcolor[rgb]{0.25,0.44,0.63}{{#1}}}
    \newcommand{\CommentTok}[1]{\textcolor[rgb]{0.38,0.63,0.69}{\textit{{#1}}}}
    \newcommand{\OtherTok}[1]{\textcolor[rgb]{0.00,0.44,0.13}{{#1}}}
    \newcommand{\AlertTok}[1]{\textcolor[rgb]{1.00,0.00,0.00}{\textbf{{#1}}}}
    \newcommand{\FunctionTok}[1]{\textcolor[rgb]{0.02,0.16,0.49}{{#1}}}
    \newcommand{\RegionMarkerTok}[1]{{#1}}
    \newcommand{\ErrorTok}[1]{\textcolor[rgb]{1.00,0.00,0.00}{\textbf{{#1}}}}
    \newcommand{\NormalTok}[1]{{#1}}
    
    % Additional commands for more recent versions of Pandoc
    \newcommand{\ConstantTok}[1]{\textcolor[rgb]{0.53,0.00,0.00}{{#1}}}
    \newcommand{\SpecialCharTok}[1]{\textcolor[rgb]{0.25,0.44,0.63}{{#1}}}
    \newcommand{\VerbatimStringTok}[1]{\textcolor[rgb]{0.25,0.44,0.63}{{#1}}}
    \newcommand{\SpecialStringTok}[1]{\textcolor[rgb]{0.73,0.40,0.53}{{#1}}}
    \newcommand{\ImportTok}[1]{{#1}}
    \newcommand{\DocumentationTok}[1]{\textcolor[rgb]{0.73,0.13,0.13}{\textit{{#1}}}}
    \newcommand{\AnnotationTok}[1]{\textcolor[rgb]{0.38,0.63,0.69}{\textbf{\textit{{#1}}}}}
    \newcommand{\CommentVarTok}[1]{\textcolor[rgb]{0.38,0.63,0.69}{\textbf{\textit{{#1}}}}}
    \newcommand{\VariableTok}[1]{\textcolor[rgb]{0.10,0.09,0.49}{{#1}}}
    \newcommand{\ControlFlowTok}[1]{\textcolor[rgb]{0.00,0.44,0.13}{\textbf{{#1}}}}
    \newcommand{\OperatorTok}[1]{\textcolor[rgb]{0.40,0.40,0.40}{{#1}}}
    \newcommand{\BuiltInTok}[1]{{#1}}
    \newcommand{\ExtensionTok}[1]{{#1}}
    \newcommand{\PreprocessorTok}[1]{\textcolor[rgb]{0.74,0.48,0.00}{{#1}}}
    \newcommand{\AttributeTok}[1]{\textcolor[rgb]{0.49,0.56,0.16}{{#1}}}
    \newcommand{\InformationTok}[1]{\textcolor[rgb]{0.38,0.63,0.69}{\textbf{\textit{{#1}}}}}
    \newcommand{\WarningTok}[1]{\textcolor[rgb]{0.38,0.63,0.69}{\textbf{\textit{{#1}}}}}
    
    
    % Define a nice break command that doesn't care if a line doesn't already
    % exist.
    \def\br{\hspace*{\fill} \\* }
    % Math Jax compatability definitions
    \def\gt{>}
    \def\lt{<}
    % Document parameters
    \title{e1006-L3-s2018-Operators}
    
    
    

    % Pygments definitions
    
\makeatletter
\def\PY@reset{\let\PY@it=\relax \let\PY@bf=\relax%
    \let\PY@ul=\relax \let\PY@tc=\relax%
    \let\PY@bc=\relax \let\PY@ff=\relax}
\def\PY@tok#1{\csname PY@tok@#1\endcsname}
\def\PY@toks#1+{\ifx\relax#1\empty\else%
    \PY@tok{#1}\expandafter\PY@toks\fi}
\def\PY@do#1{\PY@bc{\PY@tc{\PY@ul{%
    \PY@it{\PY@bf{\PY@ff{#1}}}}}}}
\def\PY#1#2{\PY@reset\PY@toks#1+\relax+\PY@do{#2}}

\expandafter\def\csname PY@tok@w\endcsname{\def\PY@tc##1{\textcolor[rgb]{0.73,0.73,0.73}{##1}}}
\expandafter\def\csname PY@tok@c\endcsname{\let\PY@it=\textit\def\PY@tc##1{\textcolor[rgb]{0.25,0.50,0.50}{##1}}}
\expandafter\def\csname PY@tok@cp\endcsname{\def\PY@tc##1{\textcolor[rgb]{0.74,0.48,0.00}{##1}}}
\expandafter\def\csname PY@tok@k\endcsname{\let\PY@bf=\textbf\def\PY@tc##1{\textcolor[rgb]{0.00,0.50,0.00}{##1}}}
\expandafter\def\csname PY@tok@kp\endcsname{\def\PY@tc##1{\textcolor[rgb]{0.00,0.50,0.00}{##1}}}
\expandafter\def\csname PY@tok@kt\endcsname{\def\PY@tc##1{\textcolor[rgb]{0.69,0.00,0.25}{##1}}}
\expandafter\def\csname PY@tok@o\endcsname{\def\PY@tc##1{\textcolor[rgb]{0.40,0.40,0.40}{##1}}}
\expandafter\def\csname PY@tok@ow\endcsname{\let\PY@bf=\textbf\def\PY@tc##1{\textcolor[rgb]{0.67,0.13,1.00}{##1}}}
\expandafter\def\csname PY@tok@nb\endcsname{\def\PY@tc##1{\textcolor[rgb]{0.00,0.50,0.00}{##1}}}
\expandafter\def\csname PY@tok@nf\endcsname{\def\PY@tc##1{\textcolor[rgb]{0.00,0.00,1.00}{##1}}}
\expandafter\def\csname PY@tok@nc\endcsname{\let\PY@bf=\textbf\def\PY@tc##1{\textcolor[rgb]{0.00,0.00,1.00}{##1}}}
\expandafter\def\csname PY@tok@nn\endcsname{\let\PY@bf=\textbf\def\PY@tc##1{\textcolor[rgb]{0.00,0.00,1.00}{##1}}}
\expandafter\def\csname PY@tok@ne\endcsname{\let\PY@bf=\textbf\def\PY@tc##1{\textcolor[rgb]{0.82,0.25,0.23}{##1}}}
\expandafter\def\csname PY@tok@nv\endcsname{\def\PY@tc##1{\textcolor[rgb]{0.10,0.09,0.49}{##1}}}
\expandafter\def\csname PY@tok@no\endcsname{\def\PY@tc##1{\textcolor[rgb]{0.53,0.00,0.00}{##1}}}
\expandafter\def\csname PY@tok@nl\endcsname{\def\PY@tc##1{\textcolor[rgb]{0.63,0.63,0.00}{##1}}}
\expandafter\def\csname PY@tok@ni\endcsname{\let\PY@bf=\textbf\def\PY@tc##1{\textcolor[rgb]{0.60,0.60,0.60}{##1}}}
\expandafter\def\csname PY@tok@na\endcsname{\def\PY@tc##1{\textcolor[rgb]{0.49,0.56,0.16}{##1}}}
\expandafter\def\csname PY@tok@nt\endcsname{\let\PY@bf=\textbf\def\PY@tc##1{\textcolor[rgb]{0.00,0.50,0.00}{##1}}}
\expandafter\def\csname PY@tok@nd\endcsname{\def\PY@tc##1{\textcolor[rgb]{0.67,0.13,1.00}{##1}}}
\expandafter\def\csname PY@tok@s\endcsname{\def\PY@tc##1{\textcolor[rgb]{0.73,0.13,0.13}{##1}}}
\expandafter\def\csname PY@tok@sd\endcsname{\let\PY@it=\textit\def\PY@tc##1{\textcolor[rgb]{0.73,0.13,0.13}{##1}}}
\expandafter\def\csname PY@tok@si\endcsname{\let\PY@bf=\textbf\def\PY@tc##1{\textcolor[rgb]{0.73,0.40,0.53}{##1}}}
\expandafter\def\csname PY@tok@se\endcsname{\let\PY@bf=\textbf\def\PY@tc##1{\textcolor[rgb]{0.73,0.40,0.13}{##1}}}
\expandafter\def\csname PY@tok@sr\endcsname{\def\PY@tc##1{\textcolor[rgb]{0.73,0.40,0.53}{##1}}}
\expandafter\def\csname PY@tok@ss\endcsname{\def\PY@tc##1{\textcolor[rgb]{0.10,0.09,0.49}{##1}}}
\expandafter\def\csname PY@tok@sx\endcsname{\def\PY@tc##1{\textcolor[rgb]{0.00,0.50,0.00}{##1}}}
\expandafter\def\csname PY@tok@m\endcsname{\def\PY@tc##1{\textcolor[rgb]{0.40,0.40,0.40}{##1}}}
\expandafter\def\csname PY@tok@gh\endcsname{\let\PY@bf=\textbf\def\PY@tc##1{\textcolor[rgb]{0.00,0.00,0.50}{##1}}}
\expandafter\def\csname PY@tok@gu\endcsname{\let\PY@bf=\textbf\def\PY@tc##1{\textcolor[rgb]{0.50,0.00,0.50}{##1}}}
\expandafter\def\csname PY@tok@gd\endcsname{\def\PY@tc##1{\textcolor[rgb]{0.63,0.00,0.00}{##1}}}
\expandafter\def\csname PY@tok@gi\endcsname{\def\PY@tc##1{\textcolor[rgb]{0.00,0.63,0.00}{##1}}}
\expandafter\def\csname PY@tok@gr\endcsname{\def\PY@tc##1{\textcolor[rgb]{1.00,0.00,0.00}{##1}}}
\expandafter\def\csname PY@tok@ge\endcsname{\let\PY@it=\textit}
\expandafter\def\csname PY@tok@gs\endcsname{\let\PY@bf=\textbf}
\expandafter\def\csname PY@tok@gp\endcsname{\let\PY@bf=\textbf\def\PY@tc##1{\textcolor[rgb]{0.00,0.00,0.50}{##1}}}
\expandafter\def\csname PY@tok@go\endcsname{\def\PY@tc##1{\textcolor[rgb]{0.53,0.53,0.53}{##1}}}
\expandafter\def\csname PY@tok@gt\endcsname{\def\PY@tc##1{\textcolor[rgb]{0.00,0.27,0.87}{##1}}}
\expandafter\def\csname PY@tok@err\endcsname{\def\PY@bc##1{\setlength{\fboxsep}{0pt}\fcolorbox[rgb]{1.00,0.00,0.00}{1,1,1}{\strut ##1}}}
\expandafter\def\csname PY@tok@kc\endcsname{\let\PY@bf=\textbf\def\PY@tc##1{\textcolor[rgb]{0.00,0.50,0.00}{##1}}}
\expandafter\def\csname PY@tok@kd\endcsname{\let\PY@bf=\textbf\def\PY@tc##1{\textcolor[rgb]{0.00,0.50,0.00}{##1}}}
\expandafter\def\csname PY@tok@kn\endcsname{\let\PY@bf=\textbf\def\PY@tc##1{\textcolor[rgb]{0.00,0.50,0.00}{##1}}}
\expandafter\def\csname PY@tok@kr\endcsname{\let\PY@bf=\textbf\def\PY@tc##1{\textcolor[rgb]{0.00,0.50,0.00}{##1}}}
\expandafter\def\csname PY@tok@bp\endcsname{\def\PY@tc##1{\textcolor[rgb]{0.00,0.50,0.00}{##1}}}
\expandafter\def\csname PY@tok@fm\endcsname{\def\PY@tc##1{\textcolor[rgb]{0.00,0.00,1.00}{##1}}}
\expandafter\def\csname PY@tok@vc\endcsname{\def\PY@tc##1{\textcolor[rgb]{0.10,0.09,0.49}{##1}}}
\expandafter\def\csname PY@tok@vg\endcsname{\def\PY@tc##1{\textcolor[rgb]{0.10,0.09,0.49}{##1}}}
\expandafter\def\csname PY@tok@vi\endcsname{\def\PY@tc##1{\textcolor[rgb]{0.10,0.09,0.49}{##1}}}
\expandafter\def\csname PY@tok@vm\endcsname{\def\PY@tc##1{\textcolor[rgb]{0.10,0.09,0.49}{##1}}}
\expandafter\def\csname PY@tok@sa\endcsname{\def\PY@tc##1{\textcolor[rgb]{0.73,0.13,0.13}{##1}}}
\expandafter\def\csname PY@tok@sb\endcsname{\def\PY@tc##1{\textcolor[rgb]{0.73,0.13,0.13}{##1}}}
\expandafter\def\csname PY@tok@sc\endcsname{\def\PY@tc##1{\textcolor[rgb]{0.73,0.13,0.13}{##1}}}
\expandafter\def\csname PY@tok@dl\endcsname{\def\PY@tc##1{\textcolor[rgb]{0.73,0.13,0.13}{##1}}}
\expandafter\def\csname PY@tok@s2\endcsname{\def\PY@tc##1{\textcolor[rgb]{0.73,0.13,0.13}{##1}}}
\expandafter\def\csname PY@tok@sh\endcsname{\def\PY@tc##1{\textcolor[rgb]{0.73,0.13,0.13}{##1}}}
\expandafter\def\csname PY@tok@s1\endcsname{\def\PY@tc##1{\textcolor[rgb]{0.73,0.13,0.13}{##1}}}
\expandafter\def\csname PY@tok@mb\endcsname{\def\PY@tc##1{\textcolor[rgb]{0.40,0.40,0.40}{##1}}}
\expandafter\def\csname PY@tok@mf\endcsname{\def\PY@tc##1{\textcolor[rgb]{0.40,0.40,0.40}{##1}}}
\expandafter\def\csname PY@tok@mh\endcsname{\def\PY@tc##1{\textcolor[rgb]{0.40,0.40,0.40}{##1}}}
\expandafter\def\csname PY@tok@mi\endcsname{\def\PY@tc##1{\textcolor[rgb]{0.40,0.40,0.40}{##1}}}
\expandafter\def\csname PY@tok@il\endcsname{\def\PY@tc##1{\textcolor[rgb]{0.40,0.40,0.40}{##1}}}
\expandafter\def\csname PY@tok@mo\endcsname{\def\PY@tc##1{\textcolor[rgb]{0.40,0.40,0.40}{##1}}}
\expandafter\def\csname PY@tok@ch\endcsname{\let\PY@it=\textit\def\PY@tc##1{\textcolor[rgb]{0.25,0.50,0.50}{##1}}}
\expandafter\def\csname PY@tok@cm\endcsname{\let\PY@it=\textit\def\PY@tc##1{\textcolor[rgb]{0.25,0.50,0.50}{##1}}}
\expandafter\def\csname PY@tok@cpf\endcsname{\let\PY@it=\textit\def\PY@tc##1{\textcolor[rgb]{0.25,0.50,0.50}{##1}}}
\expandafter\def\csname PY@tok@c1\endcsname{\let\PY@it=\textit\def\PY@tc##1{\textcolor[rgb]{0.25,0.50,0.50}{##1}}}
\expandafter\def\csname PY@tok@cs\endcsname{\let\PY@it=\textit\def\PY@tc##1{\textcolor[rgb]{0.25,0.50,0.50}{##1}}}

\def\PYZbs{\char`\\}
\def\PYZus{\char`\_}
\def\PYZob{\char`\{}
\def\PYZcb{\char`\}}
\def\PYZca{\char`\^}
\def\PYZam{\char`\&}
\def\PYZlt{\char`\<}
\def\PYZgt{\char`\>}
\def\PYZsh{\char`\#}
\def\PYZpc{\char`\%}
\def\PYZdl{\char`\$}
\def\PYZhy{\char`\-}
\def\PYZsq{\char`\'}
\def\PYZdq{\char`\"}
\def\PYZti{\char`\~}
% for compatibility with earlier versions
\def\PYZat{@}
\def\PYZlb{[}
\def\PYZrb{]}
\makeatother


    % Exact colors from NB
    \definecolor{incolor}{rgb}{0.0, 0.0, 0.5}
    \definecolor{outcolor}{rgb}{0.545, 0.0, 0.0}



    
    % Prevent overflowing lines due to hard-to-break entities
    \sloppy 
    % Setup hyperref package
    \hypersetup{
      breaklinks=true,  % so long urls are correctly broken across lines
      colorlinks=true,
      urlcolor=urlcolor,
      linkcolor=linkcolor,
      citecolor=citecolor,
      }
    % Slightly bigger margins than the latex defaults
    
    \geometry{verbose,tmargin=1in,bmargin=1in,lmargin=1in,rmargin=1in}
    
    

    \begin{document}
    
    
    \maketitle
    
    

    
    \section{Introduction to Computing for Engineers and Computer
ScientistsBuilt-In Types, Expressions, Operators, Testing, Control
Flow}\label{introduction-to-computing-for-engineers-and-computer-scientistsbuilt-in-types-expressions-operators-testing-control-flow}

    \subsection{Logistics}\label{logistics}

\begin{itemize}
\item
  I am meeting with the course assistants tomorrow. The CAs will post
  their office hours on Piazza in a day or two.
\item
  I delayed the due date for HW1 to 30-Jan because of the setup
  problems, and the fact that I have not coordinated with the CAs.
\item
  We seem to have two Piazza forums for the class. Please use the one
  linked from CourseWorks, which most of you seem to have been using.
  (piazza.com/columbia/spring2018/engie1006\_001\_2018\_1introtocompforengappsci)
\end{itemize}

    \subsection{Questions}\label{questions}

\subsubsection{Limit on Numbers}\label{limit-on-numbers}

After the last lecture:

\begin{itemize}
\item
  Student question -"Hey, Prof. You told us the biggest number in a
  computer is \(2^{64}.\)

  \begin{itemize}
  \tightlist
  \item
    The largest known prime number is \(2^{77,232,917} − 1\)
    (https://en.wikipedia.org/wiki/Largest\_known\_prime\_number)
  \item
    Presumably they used a computer to figure this out, how did that
    work?"
  \end{itemize}
\item
  Prof Answer: "From the Python 3 manual, 'The sys.maxint constant was
  removed, since there is no longer a limit to the value of integers.'"
\item
  Student: "Umm. How does that work?"
\item
  Prof Answer:
\end{itemize}

    \paragraph{Floating Point}\label{floating-point}

    \begin{Verbatim}[commandchars=\\\{\}]
{\color{incolor}In [{\color{incolor}49}]:} \PY{k+kn}{import} \PY{n+nn}{sys}
         \PY{n+nb}{print}\PY{p}{(}\PY{l+s+s2}{\PYZdq{}}\PY{l+s+s2}{The largest Python float possible on my system is }\PY{l+s+s2}{\PYZdq{}}\PY{p}{,} \PY{n}{sys}\PY{o}{.}\PY{n}{float\PYZus{}info}\PY{o}{.}\PY{n}{max}\PY{p}{)}
         \PY{n+nb}{print}\PY{p}{(}\PY{l+s+s2}{\PYZdq{}}\PY{l+s+s2}{The largest exponent for base 10 is  }\PY{l+s+s2}{\PYZdq{}}\PY{p}{,} \PY{n}{sys}\PY{o}{.}\PY{n}{float\PYZus{}info}\PY{o}{.}\PY{n}{max\PYZus{}10\PYZus{}exp}\PY{p}{)}
         \PY{n+nb}{print}\PY{p}{(}\PY{l+s+s2}{\PYZdq{}}\PY{l+s+s2}{Maximum number of digits of precision is }\PY{l+s+s2}{\PYZdq{}}\PY{p}{,} \PY{n}{sys}\PY{o}{.}\PY{n}{float\PYZus{}info}\PY{o}{.}\PY{n}{dig}\PY{p}{)}
\end{Verbatim}


    \begin{Verbatim}[commandchars=\\\{\}]
The largest Python float possible on my system is  1.7976931348623157e+308
The largest exponent for base 10 is   308
Maximum number of digits of precision is  15

    \end{Verbatim}

    \begin{itemize}
\item
  1.7976931348623157e+308 has has 308 decimal digits.
\item
  \(2^{1023} \approx 8.988465674312 \times 10^{307} \approx 0.8988465674312 \times 10^{308}\)
\item
  The 52-mantissa (in binary) is 1.111111 ... 111.

  \begin{itemize}
  \tightlist
  \item
    Binary \(1.111111111111111111111111\) = \(1.99999994039535522461\)
    decimal.
  \item
    So, we get
    \((1.99999994039535522461 \times 0.8988465674312) \times 10^{308} \approx 1.797693081287 \times 10^{308}\)
  \end{itemize}
\end{itemize}

\paragraph{Integers}\label{integers}

\begin{itemize}
\item
  Since we have 64 bits, it seems the largest integer could be
  \(2^{63}\) with 1 bit for sign.
\item
  This is pretty big, but ...

  \begin{itemize}
  \tightlist
  \item
    "The sys.maxint constant was removed, since there is no longer a
    limit to the value of integers."
    (https://docs.python.org/3/whatsnew/3.0.html)
  \item
    The largest known prime number is \(2^{77,232,917} − 1,\) which has
    23,249,425 digits.
  \end{itemize}
\item
  How does this work?

  \begin{itemize}
  \tightlist
  \item
    Python, and other languages' libraries, do not use the built-in
    hardware for very large numbers.
  \item
    The libraries do it symbolically using software.
  \end{itemize}
\item
  \(2^{64} \approx 1.844674407371\times10^{19}\), which has 20 digits.
\item
  What is the sum of the following 25 digit numbers?

  \begin{equation}
  1234567890123456789012345 + 10987654321098765432109875
  \end{equation}
\item
  You could do this with a pencil and piece of paper, by basically
  following a "program" you learned in school.
\item
  The computer treats very large integers as an array of digits and
  follows an algorithm similar to what you learned in school.
\end{itemize}

    \begin{Verbatim}[commandchars=\\\{\}]
{\color{incolor}In [{\color{incolor}2}]:} \PY{c+c1}{\PYZsh{} This module supports addition of large, positive integers.}
        \PY{c+c1}{\PYZsh{} An example is}
        \PY{c+c1}{\PYZsh{} 1234567890123456789012345 + 10987654321098765432109875}
        \PY{c+c1}{\PYZsh{} Both operands are larger than sys.maxsize = 9223372036854775807,}
        \PY{c+c1}{\PYZsh{} which is the largest integer the system can represent in 64 bits.}
        \PY{c+c1}{\PYZsh{}}
        \PY{c+c1}{\PYZsh{} The module symbolically does the addition they way students}
        \PY{c+c1}{\PYZsh{} learn in in school. The integer input is a text string of digits.}
        \PY{c+c1}{\PYZsh{}}
        \PY{c+c1}{\PYZsh{} The module expands each operand to a list of digits, and then adds}
        \PY{c+c1}{\PYZsh{} one position at a time, with support for carry.}
        \PY{c+c1}{\PYZsh{}}
        
        \PY{c+c1}{\PYZsh{} Input is a string of the form \PYZdq{}109876\PYZdq{}}
        \PY{c+c1}{\PYZsh{} Output is an array of the form [1, 0, 9, 8, 7, 6]}
        \PY{c+c1}{\PYZsh{} NOTE: The function should, but does not check for bad input like}
        \PY{c+c1}{\PYZsh{} non\PYZhy{}digits.}
        \PY{c+c1}{\PYZsh{}}
        \PY{k}{def} \PY{n+nf}{num\PYZus{}str\PYZus{}to\PYZus{}digits}\PY{p}{(}\PY{n}{num}\PY{p}{)}\PY{p}{:}
            \PY{c+c1}{\PYZsh{} Hold the result.}
            \PY{n}{digits} \PY{o}{=} \PY{p}{[}\PY{p}{]}
            
            \PY{c+c1}{\PYZsh{} Loop through each character in the string.}
            \PY{k}{for} \PY{n}{char} \PY{o+ow}{in} \PY{n}{num}\PY{p}{:}
                \PY{c+c1}{\PYZsh{} Convert the character in the position, e.g. \PYZsq{}3\PYZsq{}}
                \PY{c+c1}{\PYZsh{} To the integer it represents, e.g. 3}
                \PY{c+c1}{\PYZsh{} And append to the array}
                \PY{n}{digits}\PY{o}{.}\PY{n}{append}\PY{p}{(} \PY{n+nb}{int}\PY{p}{(}\PY{n}{char}\PY{p}{)} \PY{p}{)}
                
            \PY{c+c1}{\PYZsh{} return the result}
            \PY{k}{return} \PY{n}{digits}
        
        \PY{c+c1}{\PYZsh{} Reverse of num\PYZus{}str\PYZus{}to\PYZus{}digits.}
        \PY{c+c1}{\PYZsh{} Takes an input of the form [1, 2, 3, 7]}
        \PY{c+c1}{\PYZsh{} Returns \PYZdq{}1237\PYZdq{}}
        \PY{c+c1}{\PYZsh{}}
        \PY{k}{def} \PY{n+nf}{digits\PYZus{}to\PYZus{}num\PYZus{}str}\PY{p}{(}\PY{n}{d}\PY{p}{)}\PY{p}{:}
            \PY{c+c1}{\PYZsh{} map the str(x) function to each element in the list}
            \PY{c+c1}{\PYZsh{} The map converts [1, 2, 3, 7] to [\PYZdq{}1\PYZdq{}, \PYZdq{}2\PYZdq{}, \PYZdq{}3\PYZdq{}, \PYZdq{}7\PYZdq{}]}
            \PY{c+c1}{\PYZsh{} Then return a string which is the elements \PYZdq{}joined\PYZdq{}}
            \PY{n}{s} \PY{o}{=} \PY{l+s+s2}{\PYZdq{}}\PY{l+s+s2}{\PYZdq{}}\PY{o}{.}\PY{n}{join}\PY{p}{(}\PY{n+nb}{map}\PY{p}{(}\PY{n+nb}{str}\PY{p}{,}\PY{n}{d}\PY{p}{)}\PY{p}{)}
            \PY{k}{return} \PY{n}{s}
        
        \PY{c+c1}{\PYZsh{} The input is an array of digits and a position.}
        \PY{c+c1}{\PYZsh{} The position is the 10\PYZca{}i digit in the list representation}
        \PY{c+c1}{\PYZsh{} of the number, e.g. if i is 3 and num is [1, 2, 3, 4, 5] the}
        \PY{c+c1}{\PYZsh{} result is 2. If i is 0, the result is 5.}
        \PY{c+c1}{\PYZsh{}}
        \PY{c+c1}{\PYZsh{} If the 10s position is higher than the numer of digits, return 0}
        \PY{c+c1}{\PYZsh{}}
        \PY{k}{def} \PY{n+nf}{get\PYZus{}digit\PYZus{}from\PYZus{}list}\PY{p}{(}\PY{n}{num}\PY{p}{,} \PY{n}{i}\PY{p}{)}\PY{p}{:}
            \PY{n}{result} \PY{o}{=} \PY{l+m+mi}{0}
            
            \PY{c+c1}{\PYZsh{} Length of the list. Also means that the highest 10s position}
            \PY{c+c1}{\PYZsh{} is 10**(len \PYZhy{} 1)}
            \PY{c+c1}{\PYZsh{}}
            \PY{n}{length} \PY{o}{=} \PY{n+nb}{len}\PY{p}{(}\PY{n}{num}\PY{p}{)}
            
            \PY{c+c1}{\PYZsh{} We need to start at the end. So, the zero position is \PYZhy{}1,}
            \PY{c+c1}{\PYZsh{} the 10s position is \PYZhy{}2, ... We also need to check if the 10s}
            \PY{c+c1}{\PYZsh{} position is to larger.}
            \PY{n}{i} \PY{o}{=} \PY{n}{i} \PY{o}{+} \PY{l+m+mi}{1}
            \PY{k}{if} \PY{n}{i} \PY{o}{\PYZgt{}} \PY{n}{length}\PY{p}{:}
                \PY{n}{result} \PY{o}{=} \PY{l+m+mi}{0}
            \PY{k}{else}\PY{p}{:}
                \PY{n}{result} \PY{o}{=} \PY{n}{num}\PY{p}{[}\PY{o}{\PYZhy{}}\PY{l+m+mi}{1} \PY{o}{*} \PY{n}{i}\PY{p}{]}
                
            \PY{c+c1}{\PYZsh{} Return result.}
            \PY{k}{return} \PY{n}{result}
        
        
        \PY{c+c1}{\PYZsh{} Perform symbolic addition.}
        \PY{c+c1}{\PYZsh{} Input is two lists of integers symbolically representing a number.}
        \PY{c+c1}{\PYZsh{} For example, [1, 2, 0, 9] represents \PYZdq{}1209\PYZdq{}}
        \PY{c+c1}{\PYZsh{} The result is a symbolic representation of the sum. For example}
        \PY{c+c1}{\PYZsh{} [1, 2, 3, 4] + [9, 1, 1, 1] is [1,0,3,4,5]}
        \PY{c+c1}{\PYZsh{}}
        \PY{k}{def} \PY{n+nf}{add\PYZus{}num\PYZus{}arrays}\PY{p}{(}\PY{n}{n1}\PY{p}{,} \PY{n}{n2}\PY{p}{)}\PY{p}{:}
            
            \PY{c+c1}{\PYZsh{} Get length of the longest number. Addition needs to}
            \PY{c+c1}{\PYZsh{} process every decimal place.}
            \PY{n}{max\PYZus{}len} \PY{o}{=} \PY{n+nb}{max}\PY{p}{(}\PY{n+nb}{len}\PY{p}{(}\PY{n}{n1}\PY{p}{)}\PY{p}{,} \PY{n+nb}{len}\PY{p}{(}\PY{n}{n2}\PY{p}{)}\PY{p}{)}
            
            \PY{c+c1}{\PYZsh{} Placeholder for result.}
            \PY{n}{result} \PY{o}{=} \PY{p}{[}\PY{p}{]}
            
            \PY{c+c1}{\PYZsh{} Represents the 10s position we are adding, and starts at 0}
            \PY{n}{i} \PY{o}{=} \PY{l+m+mi}{0}
            \PY{c+c1}{\PYZsh{} Do we have to carry the 1 to the next position.}
            \PY{n}{carry} \PY{o}{=} \PY{l+m+mi}{0}
            
            \PY{c+c1}{\PYZsh{} Loop and examine every digit.}
            \PY{k}{while} \PY{n}{i} \PY{o}{\PYZlt{}} \PY{n}{max\PYZus{}len}\PY{p}{:}
                
                \PY{c+c1}{\PYZsh{} Get the 10**i digit from each number.}
                \PY{n}{d1} \PY{o}{=} \PY{n}{get\PYZus{}digit\PYZus{}from\PYZus{}list}\PY{p}{(}\PY{n}{n1}\PY{p}{,} \PY{n}{i}\PY{p}{)}
                \PY{n}{d2} \PY{o}{=} \PY{n}{get\PYZus{}digit\PYZus{}from\PYZus{}list}\PY{p}{(}\PY{n}{n2}\PY{p}{,} \PY{n}{i}\PY{p}{)}
                \PY{c+c1}{\PYZsh{}print(\PYZdq{}The two digits are \PYZdq{}, d1, \PYZdq{}, \PYZdq{}, d2)}
                
                \PY{c+c1}{\PYZsh{} Next pass through the loop will examine 10**(i+1)}
                \PY{c+c1}{\PYZsh{} if one of the numbers is big enough}
                \PY{n}{i} \PY{o}{=} \PY{n}{i} \PY{o}{+} \PY{l+m+mi}{1}
                
                \PY{c+c1}{\PYZsh{} The result value for 10**i is the sum of the two}
                \PY{c+c1}{\PYZsh{} digits plus anything carried.}
                \PY{n}{new\PYZus{}digit} \PY{o}{=} \PY{n}{d1} \PY{o}{+} \PY{n}{d2} \PY{o}{+} \PY{n}{carry}
                
                \PY{c+c1}{\PYZsh{} If the digit is larger than 10, subtract 10 and carry 1}
                \PY{k}{if} \PY{n}{new\PYZus{}digit} \PY{o}{\PYZgt{}}\PY{o}{=} \PY{l+m+mi}{10}\PY{p}{:}
                    \PY{n}{carry} \PY{o}{=} \PY{l+m+mi}{1}
                    \PY{n}{new\PYZus{}digit} \PY{o}{=} \PY{n}{new\PYZus{}digit} \PY{o}{\PYZhy{}} \PY{l+m+mi}{10}
                \PY{k}{else}\PY{p}{:}
                    \PY{n}{carry} \PY{o}{=} \PY{l+m+mi}{0}
                
                \PY{c+c1}{\PYZsh{} Add the digit to the result at the beginning. We compute the}
                \PY{c+c1}{\PYZsh{} result going right to left.}
                \PY{n}{result}\PY{o}{.}\PY{n}{insert}\PY{p}{(}\PY{l+m+mi}{0}\PY{p}{,}\PY{n}{new\PYZus{}digit}\PY{p}{)}
                
            \PY{c+c1}{\PYZsh{} When we have processed all digits, if we still have a carry,}
            \PY{c+c1}{\PYZsh{} then we add a 1 at the beginning.}
            \PY{k}{if} \PY{n}{carry} \PY{o}{==} \PY{l+m+mi}{1}\PY{p}{:}
                \PY{n}{result}\PY{o}{.}\PY{n}{insert}\PY{p}{(}\PY{l+m+mi}{0}\PY{p}{,}\PY{l+m+mi}{1}\PY{p}{)}
                
            \PY{k}{return} \PY{n}{result}
        
        
        \PY{c+c1}{\PYZsh{} This code below would be in a separate file that we use to test the module above.}
        \PY{c+c1}{\PYZsh{} Just easiest to put here for now.}
        \PY{c+c1}{\PYZsh{}}
        \PY{n}{num1} \PY{o}{=} \PY{l+s+s2}{\PYZdq{}}\PY{l+s+s2}{1234567890123456789012345}\PY{l+s+s2}{\PYZdq{}}
        \PY{n}{num2} \PY{o}{=} \PY{l+s+s2}{\PYZdq{}}\PY{l+s+s2}{10987654321098765432109875}\PY{l+s+s2}{\PYZdq{}}
        \PY{n}{num3} \PY{o}{=} \PY{l+s+s2}{\PYZdq{}}\PY{l+s+s2}{8234}\PY{l+s+s2}{\PYZdq{}}
        \PY{n}{num4} \PY{o}{=} \PY{l+s+s2}{\PYZdq{}}\PY{l+s+s2}{1906}\PY{l+s+s2}{\PYZdq{}}
        
        \PY{n}{dig1} \PY{o}{=} \PY{k+kc}{None}
        \PY{n}{dig2} \PY{o}{=} \PY{k+kc}{None}
        \PY{n}{dig3} \PY{o}{=} \PY{k+kc}{None}
            
        \PY{k}{def} \PY{n+nf}{test\PYZus{}digits\PYZus{}to\PYZus{}num\PYZus{}str}\PY{p}{(}\PY{n}{d}\PY{p}{)}\PY{p}{:}
            \PY{n}{s} \PY{o}{=} \PY{n}{digits\PYZus{}to\PYZus{}num\PYZus{}str}\PY{p}{(}\PY{n}{d}\PY{p}{)}
            \PY{n+nb}{print}\PY{p}{(}\PY{l+s+s2}{\PYZdq{}}\PY{l+s+s2}{Coverting }\PY{l+s+s2}{\PYZdq{}}\PY{p}{,} \PY{n}{d}\PY{p}{,} \PY{l+s+s2}{\PYZdq{}}\PY{l+s+s2}{to a num string is }\PY{l+s+s2}{\PYZdq{}}\PY{p}{,} \PY{n}{s}\PY{p}{)}
            
        \PY{k}{def} \PY{n+nf}{test\PYZus{}string\PYZus{}to\PYZus{}array}\PY{p}{(}\PY{p}{)}\PY{p}{:}
            \PY{k}{global} \PY{n}{dig1}
            \PY{k}{global} \PY{n}{dig2}
            \PY{k}{global} \PY{n}{dig3}
            \PY{n}{dig1} \PY{o}{=} \PY{n}{num\PYZus{}str\PYZus{}to\PYZus{}digits}\PY{p}{(}\PY{n}{num1}\PY{p}{)}
            \PY{n+nb}{print}\PY{p}{(}\PY{l+s+s2}{\PYZdq{}}\PY{l+s+s2}{dig1 = }\PY{l+s+s2}{\PYZdq{}}\PY{p}{,} \PY{n}{dig1}\PY{p}{)}
            \PY{n}{dig2} \PY{o}{=} \PY{n}{num\PYZus{}str\PYZus{}to\PYZus{}digits}\PY{p}{(}\PY{n}{num2}\PY{p}{)}
            \PY{n+nb}{print}\PY{p}{(}\PY{l+s+s2}{\PYZdq{}}\PY{l+s+s2}{dig2 = }\PY{l+s+s2}{\PYZdq{}}\PY{p}{,} \PY{n}{dig2}\PY{p}{)}
            \PY{n}{dig3} \PY{o}{=} \PY{n}{num\PYZus{}str\PYZus{}to\PYZus{}digits}\PY{p}{(}\PY{n}{num3}\PY{p}{)}
            \PY{n+nb}{print}\PY{p}{(}\PY{l+s+s2}{\PYZdq{}}\PY{l+s+s2}{dig3 = }\PY{l+s+s2}{\PYZdq{}}\PY{p}{,} \PY{n}{dig3}\PY{p}{)}
        
        \PY{k}{def} \PY{n+nf}{test\PYZus{}get\PYZus{}digit\PYZus{}from\PYZus{}list}\PY{p}{(}\PY{p}{)}\PY{p}{:}
            \PY{n}{exp} \PY{o}{=} \PY{l+m+mi}{0}
            \PY{n+nb}{print}\PY{p}{(}\PY{l+s+s2}{\PYZdq{}}\PY{l+s+s2}{10 to }\PY{l+s+s2}{\PYZdq{}}\PY{p}{,} \PY{n}{exp}\PY{p}{,} \PY{l+s+s2}{\PYZdq{}}\PY{l+s+s2}{ place digit for }\PY{l+s+s2}{\PYZdq{}}\PY{p}{,} \PY{n}{dig3}\PY{p}{,} \PY{l+s+s2}{\PYZdq{}}\PY{l+s+s2}{ is }\PY{l+s+s2}{\PYZdq{}}\PY{p}{,} \PY{n}{get\PYZus{}digit\PYZus{}from\PYZus{}list}\PY{p}{(}\PY{n}{dig3}\PY{p}{,} \PY{n}{exp}\PY{p}{)}\PY{p}{)}
            \PY{n}{exp} \PY{o}{=} \PY{l+m+mi}{3}
            \PY{n+nb}{print}\PY{p}{(}\PY{l+s+s2}{\PYZdq{}}\PY{l+s+s2}{10 to }\PY{l+s+s2}{\PYZdq{}}\PY{p}{,} \PY{n}{exp}\PY{p}{,} \PY{l+s+s2}{\PYZdq{}}\PY{l+s+s2}{ place digit for }\PY{l+s+s2}{\PYZdq{}}\PY{p}{,} \PY{n}{dig3}\PY{p}{,} \PY{l+s+s2}{\PYZdq{}}\PY{l+s+s2}{ is }\PY{l+s+s2}{\PYZdq{}}\PY{p}{,} \PY{n}{get\PYZus{}digit\PYZus{}from\PYZus{}list}\PY{p}{(}\PY{n}{dig3}\PY{p}{,} \PY{n}{exp}\PY{p}{)}\PY{p}{)}
            \PY{n}{exp} \PY{o}{=} \PY{l+m+mi}{25}
            \PY{n+nb}{print}\PY{p}{(}\PY{l+s+s2}{\PYZdq{}}\PY{l+s+s2}{10 to }\PY{l+s+s2}{\PYZdq{}}\PY{p}{,} \PY{n}{exp}\PY{p}{,} \PY{l+s+s2}{\PYZdq{}}\PY{l+s+s2}{ place digit for }\PY{l+s+s2}{\PYZdq{}}\PY{p}{,} \PY{n}{dig3}\PY{p}{,} \PY{l+s+s2}{\PYZdq{}}\PY{l+s+s2}{ is }\PY{l+s+s2}{\PYZdq{}}\PY{p}{,} \PY{n}{get\PYZus{}digit\PYZus{}from\PYZus{}list}\PY{p}{(}\PY{n}{dig3}\PY{p}{,} \PY{n}{exp}\PY{p}{)}\PY{p}{)}
            
        \PY{n}{test\PYZus{}string\PYZus{}to\PYZus{}array}\PY{p}{(}\PY{p}{)}
        \PY{n}{test\PYZus{}get\PYZus{}digit\PYZus{}from\PYZus{}list}\PY{p}{(}\PY{p}{)}
        
        \PY{n}{dig1} \PY{o}{=} \PY{n}{num\PYZus{}str\PYZus{}to\PYZus{}digits}\PY{p}{(}\PY{n}{num1}\PY{p}{)}
        \PY{n}{dig2} \PY{o}{=} \PY{n}{num\PYZus{}str\PYZus{}to\PYZus{}digits}\PY{p}{(}\PY{n}{num2}\PY{p}{)}
        
        \PY{n}{test\PYZus{}digits\PYZus{}to\PYZus{}num\PYZus{}str}\PY{p}{(}\PY{n}{dig1}\PY{p}{)}
        
        \PY{n}{dig3} \PY{o}{=} \PY{n}{num\PYZus{}str\PYZus{}to\PYZus{}digits}\PY{p}{(}\PY{n}{num3}\PY{p}{)}
        \PY{n}{dig4} \PY{o}{=} \PY{n}{num\PYZus{}str\PYZus{}to\PYZus{}digits}\PY{p}{(}\PY{n}{num4}\PY{p}{)}
        
        \PY{n}{r} \PY{o}{=} \PY{n}{add\PYZus{}num\PYZus{}arrays}\PY{p}{(}\PY{n}{dig3}\PY{p}{,} \PY{n}{dig4}\PY{p}{)}
        \PY{n}{r} \PY{o}{=} \PY{n}{digits\PYZus{}to\PYZus{}num\PYZus{}str}\PY{p}{(}\PY{n}{r}\PY{p}{)}
        \PY{n+nb}{print} \PY{p}{(}\PY{n}{num3}\PY{p}{,} \PY{l+s+s2}{\PYZdq{}}\PY{l+s+s2}{ + }\PY{l+s+s2}{\PYZdq{}}\PY{p}{,} \PY{n}{num4}\PY{p}{,} \PY{l+s+s2}{\PYZdq{}}\PY{l+s+s2}{ = }\PY{l+s+s2}{\PYZdq{}}\PY{p}{,} \PY{n}{r}\PY{p}{)}
        
        \PY{n}{r} \PY{o}{=} \PY{n}{add\PYZus{}num\PYZus{}arrays}\PY{p}{(}\PY{n}{dig1}\PY{p}{,} \PY{n}{dig2}\PY{p}{)}
        \PY{n}{r} \PY{o}{=} \PY{n}{digits\PYZus{}to\PYZus{}num\PYZus{}str}\PY{p}{(}\PY{n}{r}\PY{p}{)}
        \PY{n+nb}{print} \PY{p}{(}\PY{n}{num1}\PY{p}{,} \PY{l+s+s2}{\PYZdq{}}\PY{l+s+s2}{ + }\PY{l+s+s2}{\PYZdq{}}\PY{p}{,} \PY{n}{num2}\PY{p}{,} \PY{l+s+s2}{\PYZdq{}}\PY{l+s+s2}{ = }\PY{l+s+s2}{\PYZdq{}}\PY{p}{,} \PY{n}{r}\PY{p}{)}
\end{Verbatim}


    \begin{Verbatim}[commandchars=\\\{\}]
dig1 =  [1, 2, 3, 4, 5, 6, 7, 8, 9, 0, 1, 2, 3, 4, 5, 6, 7, 8, 9, 0, 1, 2, 3, 4, 5]
dig2 =  [1, 0, 9, 8, 7, 6, 5, 4, 3, 2, 1, 0, 9, 8, 7, 6, 5, 4, 3, 2, 1, 0, 9, 8, 7, 5]
dig3 =  [8, 2, 3, 4]
10 to  0  place digit for  [8, 2, 3, 4]  is  4
10 to  3  place digit for  [8, 2, 3, 4]  is  8
10 to  25  place digit for  [8, 2, 3, 4]  is  0
Coverting  [1, 2, 3, 4, 5, 6, 7, 8, 9, 0, 1, 2, 3, 4, 5, 6, 7, 8, 9, 0, 1, 2, 3, 4, 5] to a num string is  1234567890123456789012345
8234  +  1906  =  10140
1234567890123456789012345  +  10987654321098765432109875  =  12222222211222222221122220

    \end{Verbatim}

    \paragraph{Comments}\label{comments}

\begin{itemize}
\item
  We just implemented a simple \emph{algorithm.}
\item
  We also learned our first principle of software engineering \(-\)
  {[}decomposition{]}(https://en.wikipedia.org/wiki/Decomposition\_(computer\_science):

  \begin{itemize}
  \tightlist
  \item
    Do not write one large program implementing all aspects of an
    algorithm.
  \item
    Break the algorithm into sub-algorithms, e.g. "How do I get the 10i
    position digit.
  \item
    Implement and test each smaller algorithm.
  \item
    Combine the smaller pieces into a solution.
  \end{itemize}
\item
  Decomposition

  \begin{itemize}
  \tightlist
  \item
    Produces more reliable software by complying with
    \href{https://en.wikipedia.org/wiki/Bounded_rationality}{bounded
    rationality}
  \item
    Improves productivity by enabling subteams to work in parallel on
    subproblems.
  \item
    Produces resuable code and modules that may apply to other
    solutions.
  \end{itemize}
\item
  A complete solution would have to implement the additional integer
  operators

  \begin{itemize}
  \tightlist
  \item
    \$- \$ is obvious
  \item
    \(p \times q,\) but multiplcation is just repetitive addition.
  \item
    \(p \div q,\) but division is just a form of multiplication.
  \item
    Boolean operators \$ =, \lt, \le, \gt, \ge.\$ How to do this should
    be clear. We just compare the individual digits and length instead
    of adding.
  \end{itemize}
\item
  You can extend the approach to floating point numbers.
\item
  Libraries that perform these function have optimizations. The
  implementation above is simple and naive.
\item
  Python uses hardware representation for small enough integers, but
  automatically converts to array like representations for larger
  integers.
\item
  There are libraries that do infinite precision math for integers and
  floats, e.g. http://mpmath.org/
\item
  \textbf{Net:}

  \begin{itemize}
  \tightlist
  \item
    Programs can symbolically implement math in software for very large
    numbers.
  \item
    The only practical limit is the total number of available memory
    bytes for numbers.
  \item
    The approach is significantly slower than direct implementation in
    hardware. Simplistically, 1 + 3 = 3

    \begin{itemize}
    \tightlist
    \item
      is one clock cycle if done in HW.
    \item
      Billions of clock cycles if done in SW.
    \end{itemize}
  \end{itemize}
\item
  I will explain HW arithmetic in a future lecture, but we have to learn
  more Python right now.
\end{itemize}

    \subsection{Python Built-In Types}\label{python-built-in-types}

\subsubsection{Overview}\label{overview}

    \subsubsection{Numeric Types}\label{numeric-types}

\begin{itemize}
\item
  Numeric types are straight forward and are what you expect.
\item
  There are a bunch of operators and functions, which we will cover
  later.
\item
  There is a very powerful built-in math library, and some even more
  powerful optional packages.
\item
  Some examples:
\end{itemize}

    \begin{Verbatim}[commandchars=\\\{\}]
{\color{incolor}In [{\color{incolor}51}]:} \PY{k+kn}{import} \PY{n+nn}{math}
         
         \PY{n}{x} \PY{o}{=} \PY{l+m+mi}{4}
         \PY{n}{y} \PY{o}{=} \PY{l+m+mf}{3.1416}
         \PY{n}{z} \PY{o}{=} \PY{n+nb}{complex}\PY{p}{(}\PY{n}{x}\PY{p}{,}\PY{n}{y}\PY{p}{)}
         \PY{n}{c} \PY{o}{=} \PY{l+m+mi}{1}\PY{o}{+}\PY{l+m+mi}{2}\PY{n}{j}
         
         \PY{n+nb}{print}\PY{p}{(}\PY{l+s+s2}{\PYZdq{}}\PY{l+s+s2}{Some basic objects:}\PY{l+s+s2}{\PYZdq{}}\PY{p}{)}
         \PY{n+nb}{print}\PY{p}{(}\PY{l+s+s2}{\PYZdq{}}\PY{l+s+s2}{x = }\PY{l+s+s2}{\PYZdq{}}\PY{p}{,} \PY{n}{x}\PY{p}{)}
         \PY{n+nb}{print}\PY{p}{(}\PY{l+s+s2}{\PYZdq{}}\PY{l+s+s2}{y = }\PY{l+s+s2}{\PYZdq{}}\PY{p}{,} \PY{n}{y}\PY{p}{)}
         \PY{n+nb}{print}\PY{p}{(}\PY{l+s+s2}{\PYZdq{}}\PY{l+s+s2}{z = }\PY{l+s+s2}{\PYZdq{}}\PY{p}{,} \PY{n}{z}\PY{p}{)}
         \PY{n+nb}{print}\PY{p}{(}\PY{l+s+s2}{\PYZdq{}}\PY{l+s+s2}{c = }\PY{l+s+s2}{\PYZdq{}}\PY{p}{,} \PY{n}{c}\PY{p}{)}
         \PY{n+nb}{print}\PY{p}{(}\PY{l+s+s2}{\PYZdq{}}\PY{l+s+se}{\PYZbs{}n}\PY{l+s+s2}{Result of operations on objects.}\PY{l+s+s2}{\PYZdq{}}\PY{p}{)}
         \PY{n+nb}{print}\PY{p}{(}\PY{l+s+s2}{\PYZdq{}}\PY{l+s+s2}{z + c }\PY{l+s+se}{\PYZbs{}t}\PY{l+s+s2}{ = }\PY{l+s+s2}{\PYZdq{}}\PY{p}{,} \PY{n}{z}\PY{o}{+}\PY{n}{c}\PY{p}{)}
         \PY{n+nb}{print}\PY{p}{(}\PY{l+s+s2}{\PYZdq{}}\PY{l+s+s2}{x + y }\PY{l+s+se}{\PYZbs{}t}\PY{l+s+s2}{ = }\PY{l+s+s2}{\PYZdq{}}\PY{p}{,} \PY{n}{x} \PY{o}{+} \PY{n}{y}\PY{p}{)}
         \PY{n+nb}{print}\PY{p}{(}\PY{l+s+s2}{\PYZdq{}}\PY{l+s+s2}{x * c }\PY{l+s+se}{\PYZbs{}t}\PY{l+s+s2}{ = }\PY{l+s+s2}{\PYZdq{}}\PY{p}{,} \PY{n}{x}\PY{o}{*}\PY{n}{c}\PY{p}{)}
         \PY{n+nb}{print}\PY{p}{(}\PY{l+s+s2}{\PYZdq{}}\PY{l+s+se}{\PYZbs{}n}\PY{l+s+s2}{Result of some more operations on objects.}\PY{l+s+s2}{\PYZdq{}}\PY{p}{)}
         \PY{n+nb}{print}\PY{p}{(}\PY{l+s+s2}{\PYZdq{}}\PY{l+s+s2}{The real part of z is }\PY{l+s+s2}{\PYZdq{}}\PY{p}{,} \PY{n}{z}\PY{o}{.}\PY{n}{real}\PY{p}{)}
         \PY{n+nb}{print}\PY{p}{(}\PY{l+s+s2}{\PYZdq{}}\PY{l+s+s2}{The imaginary part of z  is }\PY{l+s+s2}{\PYZdq{}}\PY{p}{,} \PY{n}{z}\PY{o}{.}\PY{n}{imag}\PY{p}{)}
         \PY{n+nb}{print}\PY{p}{(}\PY{l+s+s2}{\PYZdq{}}\PY{l+s+s2}{The integer part of y is }\PY{l+s+s2}{\PYZdq{}}\PY{p}{,} \PY{n+nb}{int}\PY{p}{(}\PY{n}{y}\PY{p}{)}\PY{p}{)}
         \PY{n+nb}{print}\PY{p}{(}\PY{l+s+s2}{\PYZdq{}}\PY{l+s+s2}{I can also compute the ceiling of y, which is }\PY{l+s+s2}{\PYZdq{}}\PY{p}{,} \PY{n}{math}\PY{o}{.}\PY{n}{ceil}\PY{p}{(}\PY{n}{y}\PY{p}{)}\PY{p}{)}
         \PY{n+nb}{print}\PY{p}{(}\PY{l+s+s2}{\PYZdq{}}\PY{l+s+s2}{x / y = }\PY{l+s+s2}{\PYZdq{}}\PY{p}{,} \PY{n}{x} \PY{o}{/} \PY{n}{y}\PY{p}{)}
         \PY{n+nb}{print}\PY{p}{(}\PY{l+s+s2}{\PYZdq{}}\PY{l+s+s2}{x // y =}\PY{l+s+s2}{\PYZdq{}}\PY{p}{,} \PY{n}{x} \PY{o}{/}\PY{o}{/}\PY{n}{y}\PY{p}{)}
\end{Verbatim}


    \begin{Verbatim}[commandchars=\\\{\}]
Some basic objects:
x =  4
y =  3.1416
z =  (4+3.1416j)
c =  (1+2j)

Result of operations on objects.
z + c 	 =  (5+5.1416j)
x + y 	 =  7.1416
x * c 	 =  (4+8j)

Result of some more operations on objects.
The real part of z is  4.0
The imaginary part of z  is  3.1416
The integer part of y is  3
I can also compute the ceiling of y, which is  4
x / y =  1.2732365673542145
x // y = 1.0

    \end{Verbatim}

    \begin{itemize}
\tightlist
\item
  Representing numberic types

  \begin{itemize}
  \tightlist
  \item
    int/Integer has 4 representations in Python

    \begin{itemize}
    \tightlist
    \item
      Decimal:

      \begin{itemize}
      \tightlist
      \item
        Digits are 0,1,2,3,4,5,6,7,8,9
      \item
        Examples are: 0, 231, 3978
      \end{itemize}
    \item
      \href{https://en.wikipedia.org/wiki/Hexadecimal}{Hexadecimal}:

      \begin{itemize}
      \tightlist
      \item
        Digits are 0,1,2,3,4,5,6,7,8,9,A,B,C,D,E,F
      \item
        Examples:

        \begin{itemize}
        \tightlist
        \item
          0x23 = (2 * 16) + 3 = 35
        \item
          0xFA2 = (15 * (16 * 16) + (10 * 16) + 2 = 4002
        \end{itemize}
      \end{itemize}
    \item
      \href{https://en.wikipedia.org/wiki/Octal}{Octal}

      \begin{itemize}
      \tightlist
      \item
        Digits are 0,1,2,3,4,5,6,7
      \item
        Examples:

        \begin{itemize}
        \tightlist
        \item
          0o21 = (2 * 8) + 1 = 17
        \item
          0o1001 = 513
        \end{itemize}
      \end{itemize}
    \item
      \href{https://en.wikipedia.org/wiki/Binary_number}{Binary}:

      \begin{itemize}
      \tightlist
      \item
        Digits are 0,1
      \item
        Examples:

        \begin{itemize}
        \tightlist
        \item
          0b11 = (1 * 2) + 1 = 3 -0b1000 = 8
        \end{itemize}
      \end{itemize}
    \end{itemize}
  \item
    A real number has two representations

    \begin{itemize}
    \tightlist
    \item
      \href{https://en.wikipedia.org/wiki/Decimal_floating_point}{Floating
      point decimal}:

      \begin{itemize}
      \tightlist
      \item
        Format is (decimal int).(decimal int)
      \item
        Examples:

        \begin{itemize}
        \tightlist
        \item
          3.1416
        \item
          2.7813
        \item
          1.1414
        \end{itemize}
      \end{itemize}
    \item
      \href{https://en.wikipedia.org/wiki/Scientific_notation}{Exponential
      notation}: Also known as scientific notation

      \begin{itemize}
      \tightlist
      \item
        Format is (floating point)e(int)
      \item
        Examples:

        \begin{itemize}
        \tightlist
        \item
          21e3 = 21000
        \item
          0.000213e3 = 0.213
        \end{itemize}
      \end{itemize}
    \item
      \href{https://en.wikipedia.org/wiki/Complex_number}{Complex}:

      \begin{itemize}
      \tightlist
      \item
        Format is (floating point)+(floating point)j
      \item
        Examples:

        \begin{itemize}
        \tightlist
        \item
          2.1718+3.1416j = (2.1718+3.1416j)
        \item
          complex(3,21.3) = (3+21.3j)
        \end{itemize}
      \end{itemize}
    \end{itemize}
  \end{itemize}
\item
  Limits

  \begin{itemize}
  \tightlist
  \item
    Numbers in the real world can be -infinity to infinity, and have
    infinite number of decimal places.
  \item
    The basic types (classes) in Python have upper and lower bounds.
  \item
    There are modules that support increased limits and precision.
  \end{itemize}
\end{itemize}

\begin{itemize}
\tightlist
\item
  Rounding errors

  \begin{itemize}
  \tightlist
  \item
    In base 10, there is no decimal representation for irrational
    numbers, e.g. 1/3 or sqrt(2).
  \item
    Some numbers are irrational in base 2, which is what a computer
    uses, e.g. 0.1 So, (0.1 + 0.1 + 0.1) = 0.30000000000000004, and does
    not equal 0.3
  \item
    This can break some fundamental laws of mathematics (algebra), e.g.

    \begin{itemize}
    \tightlist
    \item
      math.sqrt(2) * math.sqrt(2) = 2.0000000000000004 and not 2.
    \item
      0.3 - 0.1 - 0.1 - 0.1 = -2.7755575615628914e-17, not 0
    \item
      (a + b) + c may or may not equal a + (b + c)
    \end{itemize}
  \item
    This is a behavior of ALL computers and programming languages.
  \item
    In Python, you can handle with round()

    \begin{itemize}
    \tightlist
    \item
      0.1 + 0.1 + 0.1 does not equal 0.3
    \item
      round(0.1 + 0.1 + 0.1) DOES equal 0.3
    \end{itemize}
  \end{itemize}
\end{itemize}

    \subsubsection{Sequency Types}\label{sequency-types}

\paragraph{Overview and Text
Sequences}\label{overview-and-text-sequences}

The object type \emph{str}, or text sequence, is an example of the basic
sequence pattern - There is an object of type "sequence."

\begin{itemize}
\item
  In the example, the name of the object is \emph{t}
\item
  The object \emph{t} is itself made up of an iterable sequence of
  objects. The name of each object is logically \emph{t{[}i{]}}, e.g.

  \begin{itemize}
  \tightlist
  \item
    t{[}0{]}
  \item
    t{[}1{]}
  \item
    t{[}2{]}
  \item
    ... ...
  \end{itemize}
\item
  A text sequence/str is simply a sequence in which each t{[}i{]} is a
  "text character."
\item
  Remember,

  \begin{itemize}
  \tightlist
  \item
    In a computer, everything is a binary number.
  \item
    For text and characters, the \emph{encoding} determines the mapping
    from integer value to printable character value.
  \item
    \href{https://en.wikipedia.org/wiki/ASCII}{ASCII} (American Standard
    Code for Information Interchange) used to be very common, but was
    "American" centric.

    \begin{itemize}
    \tightlist
    \item
      An ASCII only encodes 127 characters, and comes from the teletype
      days.
    \item
      ASCII does support non-English characters.
    \end{itemize}
  \item
    IBM and some other systems used
    \href{https://en.wikipedia.org/wiki/EBCDIC}{Extended Binary Coded
    Decimal Interchange Code (EBCIDIC)} for a long time.\\
  \item
    \href{https://en.wikipedia.org/wiki/UTF-8}{UTF-8} is now the most
    common encoding.

    \begin{itemize}
    \tightlist
    \item
      UTF-8 encodes 1,112,064 characters from
      \href{https://en.wikipedia.org/wiki/Unicode}{Unicode}.
    \item
      The first 127 numbers of ASCII and UTF-8 represent the same
      characters, that is UTF-8 is a superset of ASCII.
    \end{itemize}
  \end{itemize}
\end{itemize}

\emph{Some initial fun with encodings} 

    \begin{Verbatim}[commandchars=\\\{\}]
{\color{incolor}In [{\color{incolor}52}]:} \PY{n+nb}{print}\PY{p}{(}\PY{l+s+s2}{\PYZdq{}}\PY{l+s+s2}{UTF (unicode) character number 20516 is = }\PY{l+s+s2}{\PYZdq{}}\PY{p}{,} \PY{n+nb}{chr}\PY{p}{(}\PY{l+m+mi}{20516}\PY{p}{)}\PY{p}{)}
         \PY{n+nb}{print}\PY{p}{(}\PY{l+s+s2}{\PYZdq{}}\PY{l+s+s2}{There is no way to do this because ASCII is 1 byte and is 0 to 255}\PY{l+s+s2}{\PYZdq{}}\PY{p}{)}
         \PY{n+nb}{print}\PY{p}{(}\PY{l+s+s2}{\PYZdq{}}\PY{l+s+s2}{ASCII is perfectly OK with 85 representing }\PY{l+s+s2}{\PYZdq{}}\PY{p}{,} \PY{n+nb}{chr}\PY{p}{(}\PY{l+m+mi}{85}\PY{p}{)}\PY{p}{)}
\end{Verbatim}


    \begin{Verbatim}[commandchars=\\\{\}]
UTF (unicode) character number 20516 is =  値
There is no way to do this because ASCII is 1 byte and is 0 to 255
ASCII is perfectly OK with 85 representing  U

    \end{Verbatim}

     \emph{Some function with text sequences (strings)} 

    \begin{Verbatim}[commandchars=\\\{\}]
{\color{incolor}In [{\color{incolor}53}]:} \PY{n}{mm} \PY{o}{=} \PY{l+s+s2}{\PYZdq{}}\PY{l+s+s2}{A simple string}\PY{l+s+s2}{\PYZdq{}}
         
         \PY{c+c1}{\PYZsh{} An sequence of bytes, which must be 0 .. 255}
         \PY{n}{bb} \PY{o}{=} \PY{n+nb}{bytes}\PY{p}{(}\PY{p}{[}\PY{l+m+mi}{87}\PY{p}{,} \PY{l+m+mi}{9}\PY{p}{,} \PY{l+m+mi}{88}\PY{p}{,} \PY{l+m+mi}{32}\PY{p}{,} \PY{l+m+mi}{89}\PY{p}{]}\PY{p}{)}
         \PY{c+c1}{\PYZsh{} A sequence of integers.}
         \PY{n}{tt} \PY{o}{=} \PY{p}{[}\PY{l+m+mi}{87}\PY{p}{,} \PY{l+m+mi}{9}\PY{p}{,} \PY{l+m+mi}{88}\PY{p}{,} \PY{l+m+mi}{32}\PY{p}{,} \PY{l+m+mi}{89}\PY{p}{,} \PY{l+m+mi}{20516}\PY{p}{,} \PY{l+m+mi}{1023}\PY{p}{]}
         
         \PY{n+nb}{print}\PY{p}{(}\PY{l+s+s2}{\PYZdq{}}\PY{l+s+se}{\PYZbs{}n}\PY{l+s+s2}{\PYZdq{}}\PY{p}{)}
         \PY{n+nb}{print}\PY{p}{(}\PY{l+s+s2}{\PYZdq{}}\PY{l+s+s2}{mm = }\PY{l+s+s2}{\PYZdq{}}\PY{p}{,} \PY{n}{mm}\PY{p}{)}
         \PY{n+nb}{print}\PY{p}{(}\PY{l+s+s2}{\PYZdq{}}\PY{l+s+s2}{mm as a byte sequence = }\PY{l+s+s2}{\PYZdq{}}\PY{p}{,} \PY{n+nb}{bytes}\PY{p}{(}\PY{n}{mm}\PY{p}{,}\PY{l+s+s2}{\PYZdq{}}\PY{l+s+s2}{ascii}\PY{l+s+s2}{\PYZdq{}}\PY{p}{)}\PY{p}{)}
         \PY{n+nb}{print}\PY{p}{(}\PY{l+s+s2}{\PYZdq{}}\PY{l+s+se}{\PYZbs{}n}\PY{l+s+s2}{\PYZdq{}}\PY{p}{)}
         \PY{n+nb}{print}\PY{p}{(}\PY{l+s+s2}{\PYZdq{}}\PY{l+s+s2}{bb as a byte array = }\PY{l+s+s2}{\PYZdq{}}\PY{p}{,} \PY{n}{bb}\PY{p}{)}
         \PY{n+nb}{print}\PY{p}{(}\PY{l+s+s1}{\PYZsq{}}\PY{l+s+s1}{bb as a string = }\PY{l+s+s1}{\PYZsq{}}\PY{p}{,} \PY{n}{bb}\PY{o}{.}\PY{n}{decode}\PY{p}{(}\PY{l+s+s2}{\PYZdq{}}\PY{l+s+s2}{ascii}\PY{l+s+s2}{\PYZdq{}}\PY{p}{)}\PY{p}{)}
         \PY{n+nb}{print}\PY{p}{(}\PY{l+s+s1}{\PYZsq{}}\PY{l+s+s1}{bb as a UTF string = }\PY{l+s+s1}{\PYZsq{}}\PY{p}{,} \PY{n}{bb}\PY{o}{.}\PY{n}{decode}\PY{p}{(}\PY{l+s+s2}{\PYZdq{}}\PY{l+s+s2}{utf8}\PY{l+s+s2}{\PYZdq{}}\PY{p}{)}\PY{p}{)}
         \PY{n+nb}{print}\PY{p}{(}\PY{l+s+s2}{\PYZdq{}}\PY{l+s+se}{\PYZbs{}n}\PY{l+s+s2}{\PYZdq{}}\PY{p}{)}
         \PY{n+nb}{print}\PY{p}{(}\PY{l+s+s2}{\PYZdq{}}\PY{l+s+s2}{Let}\PY{l+s+s2}{\PYZsq{}}\PY{l+s+s2}{s access parts of things.}\PY{l+s+s2}{\PYZdq{}}\PY{p}{)}
         \PY{n+nb}{print}\PY{p}{(}\PY{l+s+s2}{\PYZdq{}}\PY{l+s+s2}{The fourth thing in mm is mm[3] which is }\PY{l+s+s2}{\PYZdq{}}\PY{p}{,} \PY{n}{mm}\PY{p}{[}\PY{l+m+mi}{3}\PY{p}{]}\PY{p}{)}
         \PY{n+nb}{print}\PY{p}{(}\PY{l+s+s1}{\PYZsq{}}\PY{l+s+s1}{bb as a string = }\PY{l+s+s1}{\PYZsq{}}\PY{p}{,} \PY{n}{bb}\PY{o}{.}\PY{n}{decode}\PY{p}{(}\PY{l+s+s2}{\PYZdq{}}\PY{l+s+s2}{ascii}\PY{l+s+s2}{\PYZdq{}}\PY{p}{)}\PY{p}{,} \PY{l+s+s2}{\PYZdq{}}\PY{l+s+s2}{ and the fourth byte is }\PY{l+s+s2}{\PYZdq{}}\PY{p}{,} \PY{n}{bb}\PY{p}{[}\PY{l+m+mi}{3}\PY{p}{]}\PY{p}{)}
         \PY{n+nb}{print}\PY{p}{(}\PY{l+s+s1}{\PYZsq{}}\PY{l+s+s1}{bb as a string = }\PY{l+s+s1}{\PYZsq{}}\PY{p}{,} \PY{n}{bb}\PY{o}{.}\PY{n}{decode}\PY{p}{(}\PY{l+s+s2}{\PYZdq{}}\PY{l+s+s2}{ascii}\PY{l+s+s2}{\PYZdq{}}\PY{p}{)}\PY{p}{,} \PY{l+s+s2}{\PYZdq{}}\PY{l+s+s2}{ and the second byte is }\PY{l+s+s2}{\PYZdq{}}\PY{p}{,} \PY{n}{bb}\PY{p}{[}\PY{l+m+mi}{1}\PY{p}{]}\PY{p}{)}
         \PY{n+nb}{print}\PY{p}{(}\PY{l+s+s2}{\PYZdq{}}\PY{l+s+s2}{bb[1] is the integer }\PY{l+s+s2}{\PYZdq{}}\PY{p}{,} \PY{n}{bb}\PY{p}{[}\PY{l+m+mi}{1}\PY{p}{]}\PY{p}{,} \PY{l+s+s2}{\PYZdq{}}\PY{l+s+s2}{ which is the tab character.}\PY{l+s+s2}{\PYZdq{}} \PY{p}{)}
         
         \PY{n+nb}{print}\PY{p}{(}\PY{l+s+s2}{\PYZdq{}}\PY{l+s+se}{\PYZbs{}n}\PY{l+s+s2}{Now some fun with sequence of integers}\PY{l+s+s2}{\PYZdq{}}\PY{p}{)}
         \PY{n+nb}{print}\PY{p}{(}\PY{l+s+s2}{\PYZdq{}}\PY{l+s+s2}{tt = }\PY{l+s+s2}{\PYZdq{}}\PY{p}{,} \PY{n}{tt}\PY{p}{)}
         \PY{n+nb}{print}\PY{p}{(}\PY{l+s+s2}{\PYZdq{}}\PY{l+s+s2}{Let}\PY{l+s+s2}{\PYZsq{}}\PY{l+s+s2}{s try converting tt to a string. str(tt) = }\PY{l+s+s2}{\PYZdq{}}\PY{p}{,} \PY{n+nb}{str}\PY{p}{(}\PY{n}{tt}\PY{p}{)}\PY{p}{)}
         \PY{n+nb}{print}\PY{p}{(}\PY{l+s+s2}{\PYZdq{}}\PY{l+s+s2}{I got a string representation of the list.}\PY{l+s+s2}{\PYZdq{}}\PY{p}{)}
         \PY{n+nb}{print}\PY{p}{(}\PY{l+s+s2}{\PYZdq{}}\PY{l+s+s2}{How about a string (text sequence) which converted each int to a UTF\PYZhy{}8 chr?}\PY{l+s+s2}{\PYZdq{}}\PY{p}{)}
         \PY{n+nb}{print}\PY{p}{(}\PY{l+s+s2}{\PYZdq{}}\PY{l+s+s2}{The answer is }\PY{l+s+s2}{\PYZsq{}}\PY{l+s+s2}{\PYZsq{}}\PY{l+s+s2}{.join(map(chr,tt))}\PY{l+s+s2}{\PYZdq{}}\PY{p}{,} \PY{l+s+s1}{\PYZsq{}}\PY{l+s+s1}{\PYZsq{}}\PY{o}{.}\PY{n}{join}\PY{p}{(}\PY{n+nb}{map}\PY{p}{(}\PY{n+nb}{chr}\PY{p}{,}\PY{n}{tt}\PY{p}{)}\PY{p}{)}\PY{p}{,} \PY{l+s+s2}{\PYZdq{}}\PY{l+s+se}{\PYZbs{}n}\PY{l+s+s2}{and we will understand later.}\PY{l+s+s2}{\PYZdq{}}\PY{p}{)}
\end{Verbatim}


    \begin{Verbatim}[commandchars=\\\{\}]


mm =  A simple string
mm as a byte sequence =  b'A simple string'


bb as a byte array =  b'W\textbackslash{}tX Y'
bb as a string =  W	X Y
bb as a UTF string =  W	X Y


Let's access parts of things.
The fourth thing in mm is mm[3] which is  i
bb as a string =  W	X Y  and the fourth byte is  32
bb as a string =  W	X Y  and the second byte is  9
bb[1] is the integer  9  which is the tab character.

Now some fun with sequence of integers
tt =  [87, 9, 88, 32, 89, 20516, 1023]
Let's try converting tt to a string. str(tt) =  [87, 9, 88, 32, 89, 20516, 1023]
I got a string representation of the list.
How about a string (text sequence) which converted each int to a UTF-8 chr?
The answer is ''.join(map(chr,tt)) W	X Y値Ͽ 
and we will understand later.

    \end{Verbatim}

    \paragraph{Mutable and Immutable}\label{mutable-and-immutable}

"In object-oriented and functional programming, an \emph{immutable
object} (unchangeable object) is an object whose state cannot be
modified after it is created. This is in contrast to a \emph{mutable
object} (changeable object), which can be modified after it is created."
(https://en.wikipedia.org/wiki/Immutable\_object)

\begin{itemize}
\tightlist
\item
  Most simple objects are immutable

  \begin{itemize}
  \tightlist
  \item
    You cannot change the integer 10 to 11, 12.37 or "Hello."
  \item
    In the same way you cannot change 217345 to 341167, you cannot
    change "Hello" to "Halo."
  \end{itemize}
\item
  You can

  \begin{itemize}
  \tightlist
  \item
    Change what instance names reference.
  \item
    Make new instances using parts of immutable objects.
  \item
    etc.
  \end{itemize}
\item
  Mutability

  \begin{itemize}
  \tightlist
  \item
    Is an important concept, and its importance will become clear in
    future lectures.
  \item
    We will not focus on the concept in the early lectures or projects,
    but you may occasionally hit find something that you think should
    work, but does not.
  \end{itemize}
\end{itemize}

    \begin{Verbatim}[commandchars=\\\{\}]
{\color{incolor}In [{\color{incolor}54}]:} \PY{n}{s} \PY{o}{=} \PY{l+s+s2}{\PYZdq{}}\PY{l+s+s2}{Hello}\PY{l+s+s2}{\PYZdq{}}
         \PY{n+nb}{print} \PY{p}{(}\PY{l+s+s2}{\PYZdq{}}\PY{l+s+s2}{s = }\PY{l+s+s2}{\PYZdq{}}\PY{p}{,} \PY{n}{s}\PY{p}{)}
         
         \PY{n}{s2} \PY{o}{=} \PY{n}{s}\PY{p}{[}\PY{l+m+mi}{0}\PY{p}{]} \PY{o}{+} \PY{l+s+s2}{\PYZdq{}}\PY{l+s+s2}{a}\PY{l+s+s2}{\PYZdq{}} \PY{o}{+} \PY{n}{s}\PY{p}{[}\PY{l+m+mi}{3}\PY{p}{:}\PY{l+m+mi}{5}\PY{p}{]}
         \PY{n+nb}{print}\PY{p}{(}\PY{l+s+s2}{\PYZdq{}}\PY{l+s+s2}{s2 = }\PY{l+s+s2}{\PYZdq{}}\PY{p}{,} \PY{n}{s2}\PY{p}{)}
\end{Verbatim}


    \begin{Verbatim}[commandchars=\\\{\}]
s =  Hello
s2 =  Halo

    \end{Verbatim}

    \begin{Verbatim}[commandchars=\\\{\}]
{\color{incolor}In [{\color{incolor}55}]:} \PY{n}{s}\PY{p}{[}\PY{l+m+mi}{1}\PY{p}{]}\PY{o}{=}\PY{l+s+s1}{\PYZsq{}}\PY{l+s+s1}{a}\PY{l+s+s1}{\PYZsq{}}
\end{Verbatim}


    \begin{Verbatim}[commandchars=\\\{\}]

        ---------------------------------------------------------------------------

        TypeError                                 Traceback (most recent call last)

        <ipython-input-55-f1f3e2e8e54a> in <module>()
    ----> 1 s[1]='a'
    

        TypeError: 'str' object does not support item assignment

    \end{Verbatim}

    \paragraph{byte, bytearray, memoryview}\label{byte-bytearray-memoryview}

\begin{itemize}
\item
  \emph{Byte} is the most basic unit of computer memory.

  \begin{itemize}
  \tightlist
  \item
    A byte is 8 bits (0/1 digits)
  \item
    Decimal 0 to 255.
  \item
    Binary 00000000 to 11111111
  \item
    Hexadecimal 00,01, ... 09,0A, ..., OF, 10, 11, ..., 1F, ... FF.

    \begin{itemize}
    \tightlist
    \item
      A hex digit is 0 to F, which is 16 -\/-\textgreater{} 4 bits.
    \item
      A byte is 8 bits, or two hex digits.
    \end{itemize}
  \end{itemize}
\item
  byte is immutable. byte\_array \emph{is} mutable
\item
  We will worry about \emph{memoryview} later
\end{itemize}

    \begin{Verbatim}[commandchars=\\\{\}]
{\color{incolor}In [{\color{incolor} }]:} \PY{n}{x} \PY{o}{=} \PY{l+s+sa}{b}\PY{l+s+s1}{\PYZsq{}}\PY{l+s+s1}{El ni}\PY{l+s+se}{\PYZbs{}xc3}\PY{l+s+se}{\PYZbs{}xb1}\PY{l+s+s1}{o come camar}\PY{l+s+se}{\PYZbs{}xc3}\PY{l+s+se}{\PYZbs{}xb3}\PY{l+s+s1}{n}\PY{l+s+s1}{\PYZsq{}}
        \PY{n+nb}{print}\PY{p}{(}\PY{l+s+s2}{\PYZdq{}}\PY{l+s+s2}{x = }\PY{l+s+s2}{\PYZdq{}}\PY{p}{,}\PY{n}{x}\PY{p}{)}
        \PY{n+nb}{print}\PY{p}{(}\PY{l+s+s2}{\PYZdq{}}\PY{l+s+se}{\PYZbs{}n}\PY{l+s+s2}{x.decode(}\PY{l+s+s2}{\PYZsq{}}\PY{l+s+s2}{utf\PYZhy{}8)}\PY{l+s+s2}{\PYZsq{}}\PY{l+s+s2}{) = }\PY{l+s+s2}{\PYZdq{}}\PY{p}{,}\PY{n}{x}\PY{o}{.}\PY{n}{decode}\PY{p}{(}\PY{l+s+s2}{\PYZdq{}}\PY{l+s+s2}{utf\PYZhy{}8}\PY{l+s+s2}{\PYZdq{}}\PY{p}{)}\PY{p}{)}
        \PY{n+nb}{print}\PY{p}{(}\PY{l+s+s2}{\PYZdq{}}\PY{l+s+se}{\PYZbs{}n}\PY{l+s+s2}{x to int big endian = }\PY{l+s+s2}{\PYZdq{}}\PY{p}{,} \PY{n+nb}{int}\PY{o}{.}\PY{n}{from\PYZus{}bytes}\PY{p}{(}\PY{n}{x}\PY{p}{,}  \PY{n}{byteorder}\PY{o}{=}\PY{l+s+s1}{\PYZsq{}}\PY{l+s+s1}{big}\PY{l+s+s1}{\PYZsq{}}\PY{p}{)}\PY{p}{)}
        \PY{n+nb}{print}\PY{p}{(}\PY{l+s+s2}{\PYZdq{}}\PY{l+s+se}{\PYZbs{}n}\PY{l+s+s2}{x to int big endian = }\PY{l+s+s2}{\PYZdq{}}\PY{p}{,} \PY{n+nb}{int}\PY{o}{.}\PY{n}{from\PYZus{}bytes}\PY{p}{(}\PY{n}{x}\PY{p}{,}  \PY{n}{byteorder}\PY{o}{=}\PY{l+s+s1}{\PYZsq{}}\PY{l+s+s1}{little}\PY{l+s+s1}{\PYZsq{}}\PY{p}{)}\PY{p}{)}
        
        \PY{n}{y} \PY{o}{=} \PY{n+nb}{bytearray}\PY{p}{(}\PY{n}{x}\PY{p}{)}
        \PY{n+nb}{print}\PY{p}{(}\PY{l+s+s2}{\PYZdq{}}\PY{l+s+se}{\PYZbs{}n}\PY{l+s+s2}{y = bytearray(c) produces y =}\PY{l+s+s2}{\PYZdq{}}\PY{p}{,} \PY{n}{y}\PY{p}{)}
        \PY{n+nb}{print}\PY{p}{(}\PY{l+s+s2}{\PYZdq{}}\PY{l+s+se}{\PYZbs{}n}\PY{l+s+s2}{n ASCII is the integer }\PY{l+s+s2}{\PYZdq{}}\PY{p}{,} \PY{n+nb}{ord}\PY{p}{(}\PY{l+s+s1}{\PYZsq{}}\PY{l+s+s1}{n}\PY{l+s+s1}{\PYZsq{}}\PY{p}{)}\PY{p}{)}
        \PY{n}{y}\PY{p}{[}\PY{l+m+mi}{5}\PY{p}{]}\PY{o}{=}\PY{n+nb}{ord}\PY{p}{(}\PY{l+s+s1}{\PYZsq{}}\PY{l+s+s1}{n}\PY{l+s+s1}{\PYZsq{}}\PY{p}{)}
        \PY{n}{y}\PY{p}{[}\PY{l+m+mi}{6}\PY{p}{]}\PY{o}{=}\PY{n+nb}{ord}\PY{p}{(}\PY{l+s+s1}{\PYZsq{}}\PY{l+s+s1}{o}\PY{l+s+s1}{\PYZsq{}}\PY{p}{)}
        \PY{n}{y}\PY{p}{[}\PY{l+m+mi}{7}\PY{p}{]}\PY{o}{=}\PY{n+nb}{ord}\PY{p}{(}\PY{l+s+s1}{\PYZsq{}}\PY{l+s+s1}{ }\PY{l+s+s1}{\PYZsq{}}\PY{p}{)}
        \PY{n+nb}{print}\PY{p}{(}\PY{l+s+s2}{\PYZdq{}}\PY{l+s+s2}{I can change a bytearray. y[5]= the into ASCIii n, y[6] to ASCII o and y[7] to ASCII \PYZlt{}space\PYZgt{}, }\PY{l+s+se}{\PYZbs{}n}\PY{l+s+s2}{produces y = }\PY{l+s+s2}{\PYZdq{}}\PY{p}{,} \PY{n}{y}\PY{p}{)}
        \PY{n+nb}{print}\PY{p}{(}\PY{l+s+s2}{\PYZdq{}}\PY{l+s+s2}{\PYZdq{}}\PY{p}{)}
\end{Verbatim}


    \begin{itemize}
\tightlist
\item
  'enye' requires two bytes

  \begin{itemize}
  \tightlist
  \item
    I turned the first byte into 'n'
  \item
    And the second by to \textless{} space \textgreater{}
  \end{itemize}
\item
  Why/when would you use \emph{byte}, \emph{bytearray} and
  \emph{memoryview?}

  \begin{itemize}
  \tightlist
  \item
    Integer, float, string, ... are by far the most common. Most of our
    projects will use these basic types and the other sequence types
    (below).
  \item
    In many more advanced applications, you have to manipulate raw
    bytes, e.g.

    \begin{itemize}
    \tightlist
    \item
      Image, audio, etc. processing.
    \item
      Cryptography
    \item
      Signal processing
    \end{itemize}
  \end{itemize}
\end{itemize}

    \paragraph{Lists, Sets, Maps}\label{lists-sets-maps}

"Lists are mutable sequences, typically used to store collections of
homogeneous items (where the precise degree of similarity will vary by
application)." (https://docs.python.org/3/library/stdtypes.html\#list)

Many other languages use the term "array" for the language's similar
concept.

\textbf{A simple example}

    \begin{Verbatim}[commandchars=\\\{\}]
{\color{incolor}In [{\color{incolor} }]:} \PY{n}{x} \PY{o}{=} \PY{p}{[}\PY{l+m+mi}{1}\PY{p}{,} \PY{l+m+mi}{10}\PY{p}{,} \PY{l+m+mf}{3.1416}\PY{o}{+}\PY{l+m+mf}{2.7318}\PY{n}{j}\PY{p}{,} \PY{l+s+s2}{\PYZdq{}}\PY{l+s+s2}{Hello}\PY{l+s+s2}{\PYZdq{}}\PY{p}{,} \PY{l+s+sa}{b}\PY{l+s+s1}{\PYZsq{}}\PY{l+s+s1}{El ni}\PY{l+s+se}{\PYZbs{}xc3}\PY{l+s+se}{\PYZbs{}xb1}\PY{l+s+s1}{o come camar}\PY{l+s+se}{\PYZbs{}xc3}\PY{l+s+se}{\PYZbs{}xb3}\PY{l+s+s1}{n}\PY{l+s+s1}{\PYZsq{}}\PY{p}{]}
        \PY{n+nb}{print}\PY{p}{(}\PY{l+s+s2}{\PYZdq{}}\PY{l+s+se}{\PYZbs{}n}\PY{l+s+s2}{What is }\PY{l+s+s2}{\PYZsq{}}\PY{l+s+s2}{similar}\PY{l+s+s2}{\PYZsq{}}\PY{l+s+s2}{ about the elements of }\PY{l+s+s2}{\PYZdq{}}\PY{p}{,} \PY{n}{x}\PY{p}{,} \PY{l+s+s2}{\PYZdq{}}\PY{l+s+s2}{?}\PY{l+s+s2}{\PYZdq{}}\PY{p}{)}
        \PY{n+nb}{print}\PY{p}{(}\PY{l+s+s2}{\PYZdq{}}\PY{l+s+s2}{All are examples of built\PYZhy{}in types used in the lecture}\PY{l+s+s2}{\PYZdq{}}\PY{p}{)}
\end{Verbatim}


    \textbf{List are mutable}

    \begin{Verbatim}[commandchars=\\\{\}]
{\color{incolor}In [{\color{incolor} }]:} \PY{n}{x} \PY{o}{=} \PY{p}{[}\PY{l+m+mi}{1}\PY{p}{,} \PY{l+m+mi}{10}\PY{p}{,} \PY{l+m+mf}{3.1416}\PY{o}{+}\PY{l+m+mf}{2.7318}\PY{n}{j}\PY{p}{,} \PY{l+s+s2}{\PYZdq{}}\PY{l+s+s2}{Hello}\PY{l+s+s2}{\PYZdq{}}\PY{p}{,} \PY{l+s+sa}{b}\PY{l+s+s1}{\PYZsq{}}\PY{l+s+s1}{El ni}\PY{l+s+se}{\PYZbs{}xc3}\PY{l+s+se}{\PYZbs{}xb1}\PY{l+s+s1}{o come camar}\PY{l+s+se}{\PYZbs{}xc3}\PY{l+s+se}{\PYZbs{}xb3}\PY{l+s+s1}{n}\PY{l+s+s1}{\PYZsq{}}\PY{p}{]}
        \PY{n+nb}{print}\PY{p}{(}\PY{l+s+s2}{\PYZdq{}}\PY{l+s+se}{\PYZbs{}n}\PY{l+s+s2}{x starts out as = }\PY{l+s+s2}{\PYZdq{}}\PY{p}{,} \PY{n}{x}\PY{p}{)}
        \PY{n}{x}\PY{p}{[}\PY{l+m+mi}{1}\PY{p}{]}\PY{o}{=}\PY{n+nb}{bytes}\PY{p}{(}\PY{l+s+s2}{\PYZdq{}}\PY{l+s+s2}{Canary}\PY{l+s+s2}{\PYZdq{}}\PY{o}{.}\PY{n}{encode}\PY{p}{(}\PY{l+s+s2}{\PYZdq{}}\PY{l+s+s2}{UTF\PYZhy{}8}\PY{l+s+s2}{\PYZdq{}}\PY{p}{)}\PY{p}{)}
        \PY{n+nb}{print}\PY{p}{(}\PY{l+s+s2}{\PYZdq{}}\PY{l+s+se}{\PYZbs{}n}\PY{l+s+s2}{x becomes = }\PY{l+s+s2}{\PYZdq{}}\PY{p}{,} \PY{n}{x}\PY{p}{)}
        
        \PY{n}{y} \PY{o}{=} \PY{p}{[}\PY{l+s+s2}{\PYZdq{}}\PY{l+s+s2}{foo}\PY{l+s+s2}{\PYZdq{}}\PY{p}{,} \PY{l+m+mf}{2e9}\PY{p}{]}
        \PY{n+nb}{print}\PY{p}{(}\PY{l+s+s2}{\PYZdq{}}\PY{l+s+se}{\PYZbs{}n}\PY{l+s+s2}{y = }\PY{l+s+s2}{\PYZdq{}}\PY{p}{,} \PY{n}{y}\PY{p}{)}
        
        \PY{n}{y} \PY{o}{=} \PY{n}{x} \PY{o}{+} \PY{n}{y}
        \PY{n+nb}{print}\PY{p}{(}\PY{l+s+s2}{\PYZdq{}}\PY{l+s+se}{\PYZbs{}n}\PY{l+s+s2}{y becomes = }\PY{l+s+s2}{\PYZdq{}}\PY{p}{,} \PY{n}{y}\PY{p}{)}
        \PY{k}{del}\PY{p}{(}\PY{n}{y}\PY{p}{[}\PY{l+m+mi}{1}\PY{p}{]}\PY{p}{)}
        \PY{n+nb}{print}\PY{p}{(}\PY{l+s+s2}{\PYZdq{}}\PY{l+s+se}{\PYZbs{}n}\PY{l+s+s2}{y becomes = }\PY{l+s+s2}{\PYZdq{}}\PY{p}{,} \PY{n}{y}\PY{p}{)}
\end{Verbatim}


    \textbf{Lists are fundamental of data, analytics and plotting}

    \begin{Verbatim}[commandchars=\\\{\}]
{\color{incolor}In [{\color{incolor}4}]:} \PY{k+kn}{import} \PY{n+nn}{numpy} \PY{k}{as} \PY{n+nn}{np} 
        \PY{k+kn}{import} \PY{n+nn}{matplotlib}\PY{n+nn}{.}\PY{n+nn}{pyplot} \PY{k}{as} \PY{n+nn}{plt}
        
        
        \PY{n}{a} \PY{o}{=} \PY{n}{np}\PY{o}{.}\PY{n}{array}\PY{p}{(}\PY{p}{[}\PY{l+m+mi}{1}\PY{p}{,}\PY{l+m+mi}{3}\PY{p}{]}\PY{p}{)} 
        \PY{n+nb}{print}\PY{p}{(}\PY{l+s+s2}{\PYZdq{}}\PY{l+s+se}{\PYZbs{}n}\PY{l+s+s2}{Vector a = }\PY{l+s+s2}{\PYZdq{}}\PY{p}{,} \PY{n}{a}\PY{p}{)}
        \PY{n}{b} \PY{o}{=} \PY{n}{np}\PY{o}{.}\PY{n}{array}\PY{p}{(}\PY{p}{[}\PY{l+m+mi}{5}\PY{p}{,}\PY{l+m+mi}{2}\PY{p}{]}\PY{p}{)}
        \PY{n+nb}{print}\PY{p}{(}\PY{l+s+s2}{\PYZdq{}}\PY{l+s+s2}{Vector b = }\PY{l+s+s2}{\PYZdq{}}\PY{p}{,} \PY{n}{b}\PY{p}{)}
        
        \PY{n+nb}{print}\PY{p}{(}\PY{l+s+s2}{\PYZdq{}}\PY{l+s+se}{\PYZbs{}n}\PY{l+s+s2}{The inner product of a and b is = }\PY{l+s+s2}{\PYZdq{}}\PY{p}{,} \PY{n}{np}\PY{o}{.}\PY{n}{inner}\PY{p}{(}\PY{n}{a}\PY{p}{,}\PY{n}{b}\PY{p}{)}\PY{p}{)}
        
        \PY{n}{ad} \PY{o}{=} \PY{n}{np}\PY{o}{.}\PY{n}{add}\PY{p}{(}\PY{n}{a}\PY{p}{,}\PY{n}{b}\PY{p}{)}
        \PY{n+nb}{print}\PY{p}{(}\PY{l+s+s2}{\PYZdq{}}\PY{l+s+se}{\PYZbs{}n}\PY{l+s+s2}{The sum of a + b is the vector }\PY{l+s+s2}{\PYZdq{}}\PY{p}{,} \PY{n}{ad}\PY{p}{)}
        
        \PY{n}{ax} \PY{o}{=} \PY{n}{plt}\PY{o}{.}\PY{n}{axes}\PY{p}{(}\PY{p}{)}
        \PY{n}{plt}\PY{o}{.}\PY{n}{axis}\PY{p}{(}\PY{p}{[}\PY{l+m+mi}{0}\PY{p}{,} \PY{l+m+mi}{7}\PY{p}{,} \PY{l+m+mi}{0}\PY{p}{,} \PY{l+m+mi}{7}\PY{p}{]}\PY{p}{)}
        \PY{n}{l1} \PY{o}{=} \PY{n}{ax}\PY{o}{.}\PY{n}{arrow}\PY{p}{(}\PY{l+m+mi}{0}\PY{p}{,}\PY{l+m+mi}{0}\PY{p}{,}\PY{n}{a}\PY{p}{[}\PY{l+m+mi}{0}\PY{p}{]}\PY{p}{,}\PY{n}{a}\PY{p}{[}\PY{l+m+mi}{1}\PY{p}{]}\PY{p}{,} \PY{n}{color}\PY{o}{=}\PY{l+s+s1}{\PYZsq{}}\PY{l+s+s1}{r}\PY{l+s+s1}{\PYZsq{}}\PY{p}{,} \PY{n}{label}\PY{o}{=}\PY{l+s+s2}{\PYZdq{}}\PY{l+s+s2}{a}\PY{l+s+s2}{\PYZdq{}}\PY{p}{,}\PY{n}{head\PYZus{}width}\PY{o}{=}\PY{l+m+mf}{0.2}\PY{p}{,} \PY{n}{head\PYZus{}length}\PY{o}{=}\PY{l+m+mf}{0.2}\PY{p}{)}
        \PY{n}{l2} \PY{o}{=} \PY{n}{ax}\PY{o}{.}\PY{n}{arrow}\PY{p}{(}\PY{l+m+mi}{0}\PY{p}{,}\PY{l+m+mi}{0}\PY{p}{,} \PY{n}{b}\PY{p}{[}\PY{l+m+mi}{0}\PY{p}{]}\PY{p}{,}\PY{n}{b}\PY{p}{[}\PY{l+m+mi}{1}\PY{p}{]}\PY{p}{,} \PY{n}{color}\PY{o}{=}\PY{l+s+s1}{\PYZsq{}}\PY{l+s+s1}{b}\PY{l+s+s1}{\PYZsq{}}\PY{p}{,} \PY{n}{label}\PY{o}{=}\PY{l+s+s2}{\PYZdq{}}\PY{l+s+s2}{b}\PY{l+s+s2}{\PYZdq{}}\PY{p}{,}\PY{n}{head\PYZus{}width}\PY{o}{=}\PY{l+m+mf}{0.2}\PY{p}{,} \PY{n}{head\PYZus{}length}\PY{o}{=}\PY{l+m+mf}{0.2}\PY{p}{)}
        \PY{n}{l3} \PY{o}{=} \PY{n}{ax}\PY{o}{.}\PY{n}{arrow}\PY{p}{(}\PY{l+m+mi}{0}\PY{p}{,}\PY{l+m+mi}{0}\PY{p}{,}\PY{n}{ad}\PY{p}{[}\PY{l+m+mi}{0}\PY{p}{]}\PY{p}{,}\PY{n}{ad}\PY{p}{[}\PY{l+m+mi}{1}\PY{p}{]}\PY{p}{,} \PY{n}{color}\PY{o}{=}\PY{l+s+s1}{\PYZsq{}}\PY{l+s+s1}{g}\PY{l+s+s1}{\PYZsq{}}\PY{p}{,} \PY{n}{label}\PY{o}{=}\PY{l+s+s2}{\PYZdq{}}\PY{l+s+s2}{a+b}\PY{l+s+s2}{\PYZdq{}}\PY{p}{,}\PY{n}{head\PYZus{}width}\PY{o}{=}\PY{l+m+mf}{0.2}\PY{p}{,} \PY{n}{head\PYZus{}length}\PY{o}{=}\PY{l+m+mf}{0.2}\PY{p}{)}
        \PY{n}{l4} \PY{o}{=} \PY{n}{ax}\PY{o}{.}\PY{n}{arrow}\PY{p}{(}\PY{n}{a}\PY{p}{[}\PY{l+m+mi}{0}\PY{p}{]}\PY{p}{,}\PY{n}{a}\PY{p}{[}\PY{l+m+mi}{1}\PY{p}{]}\PY{p}{,}\PY{p}{(}\PY{n}{ad}\PY{p}{[}\PY{l+m+mi}{0}\PY{p}{]}\PY{o}{\PYZhy{}}\PY{n}{a}\PY{p}{[}\PY{l+m+mi}{0}\PY{p}{]}\PY{p}{)}\PY{p}{,}\PY{p}{(}\PY{n}{ad}\PY{p}{[}\PY{l+m+mi}{1}\PY{p}{]}\PY{o}{\PYZhy{}}\PY{n}{a}\PY{p}{[}\PY{l+m+mi}{1}\PY{p}{]}\PY{p}{)}\PY{p}{,}\PY{n}{head\PYZus{}width}\PY{o}{=}\PY{l+m+mf}{0.2}\PY{p}{,} \PY{n}{head\PYZus{}length}\PY{o}{=}\PY{l+m+mf}{0.2}\PY{p}{,}\PY{n}{color}\PY{o}{=}\PY{l+s+s1}{\PYZsq{}}\PY{l+s+s1}{c}\PY{l+s+s1}{\PYZsq{}}\PY{p}{,}\PY{n}{label}\PY{o}{=}\PY{l+s+s2}{\PYZdq{}}\PY{l+s+s2}{b starting from  a}\PY{l+s+s2}{\PYZdq{}}\PY{p}{)}
        \PY{n}{legend} \PY{o}{=} \PY{n}{ax}\PY{o}{.}\PY{n}{legend}\PY{p}{(}\PY{n}{handles}\PY{o}{=}\PY{p}{[}\PY{n}{l1}\PY{p}{,}\PY{n}{l2}\PY{p}{,} \PY{n}{l3}\PY{p}{,} \PY{n}{l4}\PY{p}{]}\PY{p}{,}\PY{n}{loc}\PY{o}{=}\PY{l+s+s1}{\PYZsq{}}\PY{l+s+s1}{upper left}\PY{l+s+s1}{\PYZsq{}}\PY{p}{)}
        
        \PY{n}{plt}\PY{o}{.}\PY{n}{show}\PY{p}{(}\PY{p}{)} 
\end{Verbatim}


    \begin{Verbatim}[commandchars=\\\{\}]

Vector a =  [1 3]
Vector b =  [5 2]

The inner product of a and b is =  11

The sum of a + b is the vector  [6 5]

    \end{Verbatim}

    \begin{center}
    \adjustimage{max size={0.9\linewidth}{0.9\paperheight}}{output_27_1.png}
    \end{center}
    { \hspace*{\fill} \\}
    
    \subsubsection{Homework 1, Part B}\label{homework-1-part-b}

\begin{itemize}
\item
  Due date, points and full description are on
  \href{https://courseworks2.columbia.edu/courses/53496/assignments/134478}{CourseWorks}
\item
  Build a Jupyter notebook with the name "\textless{} uni
  \textgreater{}\_assignment\_2"
\item
  Add the Python code from Assignment 1, Part A to a code cell in the
  Jupyter notebook.
\item
  Build your UNI

  \begin{itemize}
  \tightlist
  \item
    Create a Code cell in the notebook. Use python statements like a=23
    or b="foo".
  \item
    Assign the names last\_name, first\_name, middle\_name to the string
    values for your last, first and middle names.
  \item
    The prefix of your UNI is some combination of the first character of
    the name string. For example, my full name is "Donald Francis
    Ferguson" and my UNI prefix is "dff".
  \item
    The suffix of the UNI is a number. Assign the name uni\_suffix to
    the integer number in your UNI. Mine would be the integer 9.
  \item
    Using modifications from the code in lecture 1 and 2, create a
    string reprentation of your UNI and print it.
  \item
    Uppercase letters are fine, e.g. "DFF9" is a correct submission.
  \end{itemize}
\item
  List

  \begin{itemize}
  \tightlist
  \item
    Create a list that contains your first name, middle name, last name
    and UNI.
  \item
    Remove your middle name from the list.
  \item
    Move the UNI to the first element in the list.
  \end{itemize}
\end{itemize}

    \subsection{Operators}\label{operators}

    \subsubsection{Overview}\label{overview}

References: - Punch and Embody, section 1.7. -
\href{https://www.tutorialspoint.com/python/python_basic_operators.htm}{tutorialspoint}
is also a good overview and tutorial.

\href{}{Definitions:} "\emph{Operators} are special symbols in Python
that carry out arithmetic or logical computation. The value(s) that the
operator operates on is called the \emph{operand(s)}.

Example: 4 + 5 == 9 versus 4 + 5 = 9 - 4 + 5 == 9 - Has three operands:
4,5,9 - Two operators: +, == - o1 + o2 produces the arithmetic (or other
sum) of the operands. - o1 == o2 produces True if o1 and o2 have the
same value.

\begin{itemize}
\tightlist
\item
  4 + 5 = 9

  \begin{itemize}
  \tightlist
  \item
    Has three operands: 4,5,9
  \item
    Two operators: +, =

    \begin{itemize}
    \tightlist
    \item
      o1 + o2 produces the arithmetic (or other sum) of the operands.
    \item
      o1 = o2 sets the value of o1 to the value of o2
    \end{itemize}
  \end{itemize}
\end{itemize}

    \begin{Verbatim}[commandchars=\\\{\}]
{\color{incolor}In [{\color{incolor} }]:} \PY{l+m+mi}{4} \PY{o}{+} \PY{l+m+mi}{5}
\end{Verbatim}


    \begin{Verbatim}[commandchars=\\\{\}]
{\color{incolor}In [{\color{incolor} }]:} \PY{l+m+mi}{4} \PY{o}{+} \PY{l+m+mi}{5} \PY{o}{==} \PY{l+m+mi}{9}
\end{Verbatim}


    \begin{Verbatim}[commandchars=\\\{\}]
{\color{incolor}In [{\color{incolor} }]:} \PY{l+m+mi}{4} \PY{o}{+} \PY{l+m+mi}{5} \PY{o}{=} \PY{l+m+mi}{9}
\end{Verbatim}


    \begin{itemize}
\tightlist
\item
  Why is 4 + 5 = 9 and error?

  \begin{itemize}
  \tightlist
  \item
    4 + 5 produces 9.
  \item
    You cannot change the value of 9, even if you are trying to change
    it to 9.
  \end{itemize}
\item
  There are three kinds of operand

  \begin{itemize}
  \tightlist
  \item
    Literal: 4, 3.12, "Cat", ...
  \item
    Identifier: Think variable (name), e.g. x, radius\_str,
    circumference, ...
  \item
    Enclosure: Literals or identifier "enclosed" with (), {[}{]}, \{\},
    ...
  \end{itemize}
\item
  Python supports the following categories/kinds of operators:

  \begin{itemize}
  \tightlist
  \item
    Arithmetic Operators
  \item
    Comparison (Relational) Operators
  \item
    Assignment Operators
  \item
    Logical Operators
  \item
    Bitwise Operators
  \item
    Membership Operators
  \item
    Identity Operators
  \end{itemize}
\item
  \emph{NOTE:} Trying to follow the book. So, will cover Bitwise,
  Membership and Identity operators later.
\end{itemize}

    \subsubsection{Arithmetic Operators}\label{arithmetic-operators}

Arithmetic operators operate on numeric types: integer, float, complex.

\begin{longtable}[]{@{}c@{}}
\toprule
\tabularnewline
\midrule
\endhead
\href{https://www.tutorialspoint.com/python/python_basic_operators.htm}{\textbf{Arithmetic
Operators}}\tabularnewline
\bottomrule
\end{longtable}

    Examples: 

    Some of these operators also operate on more complex types: - The symbol
is the same. - But the operator is different (context sensitive)

    \begin{itemize}
\tightlist
\item
  In the case of o1 \textless{} operator \textgreater{} o2,

  \begin{itemize}
  \tightlist
  \item
    o1 and o1 must be compatible with respect to the operator.
  \item
    The type of o1 and o2, and the operator determine the type of the
    result.
  \end{itemize}
\end{itemize}

    \subsubsection{Comparison Operators}\label{comparison-operators}

\begin{itemize}
\tightlist
\item
  The basic format is op1 \textless{} operator \textgreater{} o2
  produces either True or False.
\end{itemize}

\begin{longtable}[]{@{}c@{}}
\toprule
\tabularnewline
\midrule
\endhead
\href{https://www.tutorialspoint.com/python/python_basic_operators.htm}{\textbf{Comparison
Operators}}\tabularnewline
\bottomrule
\end{longtable}

    \begin{itemize}
\item
  Comparison operators/expressions are most often used in control flow
  statements (covered soon).
\item
  But here is an example:
\end{itemize}

    \begin{Verbatim}[commandchars=\\\{\}]
{\color{incolor}In [{\color{incolor} }]:} \PY{k+kn}{import} \PY{n+nn}{math}
        
        \PY{k}{def} \PY{n+nf}{is\PYZus{}prime}\PY{p}{(}\PY{n}{n}\PY{p}{)}\PY{p}{:}
            
            \PY{n}{highest\PYZus{}candidate} \PY{o}{=} \PY{n}{math}\PY{o}{.}\PY{n}{floor}\PY{p}{(}\PY{n}{math}\PY{o}{.}\PY{n}{sqrt}\PY{p}{(}\PY{n}{n}\PY{p}{)}\PY{p}{)}
            \PY{n+nb}{print}\PY{p}{(}\PY{l+s+s2}{\PYZdq{}}\PY{l+s+s2}{Trying to check if }\PY{l+s+s2}{\PYZdq{}}\PY{p}{,} \PY{n}{n}\PY{p}{,} \PY{l+s+s2}{\PYZdq{}}\PY{l+s+s2}{ is prime.}\PY{l+s+s2}{\PYZdq{}}\PY{p}{)}
            \PY{n+nb}{print}\PY{p}{(}\PY{l+s+s2}{\PYZdq{}}\PY{l+s+s2}{Need to test integers from 2 ... }\PY{l+s+s2}{\PYZdq{}}\PY{p}{,} \PY{n}{highest\PYZus{}candidate}\PY{p}{)}
            
            \PY{n}{i} \PY{o}{=} \PY{l+m+mi}{2}
            \PY{n}{divisors} \PY{o}{=} \PY{k+kc}{None}
            
            \PY{c+c1}{\PYZsh{} Comparison operator}
            \PY{k}{while} \PY{p}{(}\PY{n}{i} \PY{o}{\PYZlt{}}\PY{o}{=} \PY{n}{highest\PYZus{}candidate}\PY{p}{)}\PY{p}{:}
                \PY{n}{remainder} \PY{o}{=} \PY{n}{n} \PY{o}{\PYZpc{}} \PY{n}{i}
                
                \PY{c+c1}{\PYZsh{} Comparison operator.}
                \PY{k}{if} \PY{p}{(}\PY{n}{remainder} \PY{o}{==} \PY{l+m+mi}{0}\PY{p}{)}\PY{p}{:}
                    \PY{c+c1}{\PYZsh{}print(\PYZdq{}The integer \PYZdq{}, i, \PYZdq{} divides \PYZdq{}, n)}
                    
                    \PY{c+c1}{\PYZsh{} Do not ask.}
                    \PY{k}{if} \PY{p}{(}\PY{n}{divisors} \PY{o+ow}{is} \PY{k+kc}{None}\PY{p}{)}\PY{p}{:}
                        \PY{n}{divisors} \PY{o}{=} \PY{p}{[}\PY{p}{]}
                        
                    \PY{n}{divisors}\PY{o}{.}\PY{n}{append}\PY{p}{(}\PY{n}{i}\PY{p}{)}
                    
                \PY{n}{i} \PY{o}{=} \PY{n}{i} \PY{o}{+} \PY{l+m+mi}{1}
                
            \PY{k}{return} \PY{n}{divisors}
           
            
        \PY{n+nb}{print}\PY{p}{(}\PY{l+s+s2}{\PYZdq{}}\PY{l+s+s2}{The divisors of 32 less than sqrt(32) are}\PY{l+s+s2}{\PYZdq{}}\PY{p}{,} \PY{n}{is\PYZus{}prime}\PY{p}{(}\PY{l+m+mi}{32}\PY{p}{)}\PY{p}{)}
        \PY{n+nb}{print}\PY{p}{(}\PY{l+s+s2}{\PYZdq{}}\PY{l+s+se}{\PYZbs{}n}\PY{l+s+s2}{\PYZdq{}}\PY{p}{)}
        \PY{n+nb}{print}\PY{p}{(}\PY{l+s+s2}{\PYZdq{}}\PY{l+s+s2}{The divisors of 61 less than sqrt(61) are}\PY{l+s+s2}{\PYZdq{}}\PY{p}{,} \PY{n}{is\PYZus{}prime}\PY{p}{(}\PY{l+m+mi}{61}\PY{p}{)}\PY{p}{)}
\end{Verbatim}


    \subsubsection{Assignment Operators}\label{assignment-operators}

The form is o1 \(operator\) o2, and sets o1 to the value produced by
operator and o2.

\begin{longtable}[]{@{}c@{}}
\toprule
\tabularnewline
\midrule
\endhead
\href{https://www.tutorialspoint.com/python/python_basic_operators.htm}{\textbf{Assignment
Operators}}\tabularnewline
\bottomrule
\end{longtable}

    \begin{itemize}
\tightlist
\item
  These ones drive me crazy and I never use them.
\end{itemize}

    \subsubsection{Logical Operators}\label{logical-operators}

\begin{longtable}[]{@{}c@{}}
\toprule
\tabularnewline
\midrule
\endhead
\href{https://www.tutorialspoint.com/python/python_basic_operators.htm}{\textbf{Logical
Operators}}\tabularnewline
\bottomrule
\end{longtable}

    

    \subsubsection{Summary}\label{summary}

The easiest thing to do is just play with the operators.

    \subsection{Testing and Errors}\label{testing-and-errors}

\subsubsection{Overview}\label{overview}

Punch and Embody, section 1.9, 1.10, 1.11

\textbf{Rule 5:} Test you code, often and thoroughly. (Section 1.9.1, p.
72)

Not particularly helpful. "Thank you, CPT Obvious."

    \subsubsection{Errors}\label{errors}

You will encounter three broad classes of errors. 1. Syntax: There is a
grammar error and Python will not run your program at all. 2. Runtime:
Your program is syntactically correct, and starts to run but fails. 3.
Correctness: Your program runs to completion but produces an incorrect
answer.

\textbf{Syntax Error}

The IDE typically flags these for you and will not try to run the
program.

\textbf{Runtime Error}

The program is syntactically correct, but variable values get into
states that cause errors. 1. Incompatible types for operations. 1.
Unassigned name. 1. Divide by 0. 1. etc.

    

    \textbf{An Interesting Digression}

\begin{enumerate}
\def\labelenumi{\arabic{enumi}.}
\tightlist
\item
  Let \(a\) and \(b\) be equal, non-zero quantities \(a=b\) 
\item
  Multiply both sides by \(a\) \(a^2 = ab\) 
\item
  Subtract \(b^2\) from both sides \(a^2 - b^2 = ab - b^2\) 
\item
  Do some factorization \((a + b)(a - b) = b(a - b)\) 
\item
  Divide both sides by a common value, and observe

  \begin{equation*}
  \frac{(a + b)(a - b)}{(a - b)} = \frac{b(a - b)}{(a - b)}
  \end{equation*}
\item
  Thus, \(a + b = b\) 
\item
  Since (from 1) \(a = b,\) we get
  \((a + b = a) \implies ((a + a) = a) \implies (2 x a = a) \implies (2 = 1)\)
\end{enumerate}

Computers will not make this mistake.

    \begin{Verbatim}[commandchars=\\\{\}]
{\color{incolor}In [{\color{incolor}1}]:} \PY{n}{a} \PY{o}{=} \PY{l+m+mi}{1}
        \PY{n}{b} \PY{o}{=} \PY{n}{a}
        
        \PY{p}{(}\PY{n}{a} \PY{o}{==} \PY{n}{b}\PY{p}{)}
\end{Verbatim}


\begin{Verbatim}[commandchars=\\\{\}]
{\color{outcolor}Out[{\color{outcolor}1}]:} True
\end{Verbatim}
            
    \begin{Verbatim}[commandchars=\\\{\}]
{\color{incolor}In [{\color{incolor}2}]:} \PY{n}{a}\PY{o}{*}\PY{o}{*}\PY{l+m+mi}{2} \PY{o}{==} \PY{n}{a} \PY{o}{*} \PY{n}{b}
\end{Verbatim}


\begin{Verbatim}[commandchars=\\\{\}]
{\color{outcolor}Out[{\color{outcolor}2}]:} True
\end{Verbatim}
            
    \begin{Verbatim}[commandchars=\\\{\}]
{\color{incolor}In [{\color{incolor}3}]:} \PY{n}{a}\PY{o}{*}\PY{o}{*}\PY{l+m+mi}{2} \PY{o}{\PYZhy{}} \PY{n}{b}\PY{o}{*}\PY{o}{*}\PY{l+m+mi}{2} \PY{o}{==} \PY{n}{a}\PY{o}{*}\PY{n}{b} \PY{o}{\PYZhy{}} \PY{n}{b}\PY{o}{*}\PY{o}{*}\PY{l+m+mi}{2}
\end{Verbatim}


\begin{Verbatim}[commandchars=\\\{\}]
{\color{outcolor}Out[{\color{outcolor}3}]:} True
\end{Verbatim}
            
    \begin{Verbatim}[commandchars=\\\{\}]
{\color{incolor}In [{\color{incolor}4}]:} \PY{p}{(}\PY{n}{a} \PY{o}{+} \PY{n}{b}\PY{p}{)} \PY{o}{*} \PY{p}{(}\PY{n}{a} \PY{o}{\PYZhy{}} \PY{n}{b}\PY{p}{)} \PY{o}{==} \PY{n}{b} \PY{o}{*} \PY{p}{(}\PY{n}{a} \PY{o}{\PYZhy{}} \PY{n}{b}\PY{p}{)}
\end{Verbatim}


\begin{Verbatim}[commandchars=\\\{\}]
{\color{outcolor}Out[{\color{outcolor}4}]:} True
\end{Verbatim}
            
    \begin{Verbatim}[commandchars=\\\{\}]
{\color{incolor}In [{\color{incolor}5}]:} \PY{p}{(}\PY{n}{a} \PY{o}{+} \PY{n}{b}\PY{p}{)}\PY{o}{*}\PY{p}{(}\PY{n}{a}\PY{o}{\PYZhy{}}\PY{n}{b}\PY{p}{)}\PY{o}{/}\PY{p}{(}\PY{n}{a}\PY{o}{\PYZhy{}}\PY{n}{b}\PY{p}{)}
\end{Verbatim}


    \begin{Verbatim}[commandchars=\\\{\}]

        ---------------------------------------------------------------------------

        ZeroDivisionError                         Traceback (most recent call last)

        <ipython-input-5-bd14dce8c6c1> in <module>()
    ----> 1 (a + b)*(a-b)/(a-b)
    

        ZeroDivisionError: division by zero

    \end{Verbatim}

    This formulation is pretty easy to spot, but there are much, much
trickier versions that make it hard to spot divide by 0.

    \textbf{Correctness Errors}

\begin{itemize}
\item
  Correctness errors are by far the most difficult to resolve.
\item
  The program executes but produces an incorrect result.
\item
  There are countless causes. Some examples,

  \begin{itemize}
  \tightlist
  \item
    The algorithm is incorrect.
  \item
    Mathematical formula looks correct, but parentheses and operator
    precedence are wrong.
  \item
    Incorrect assumptions: Input is in kilograms but program uses pounds
    and ounces.
  \item
    Various forms of overflow, e.g. the programmer assumed that a
    counter or value would never wrap around to 0.
  \end{itemize}
\end{itemize}

\emph{Mars Climate Orbiter}

"The Mars Climate Orbiter (formerly the Mars Surveyor '98 Orbiter) was a
338-kilogram (745 lb) robotic space probe launched by NASA on December
11, 1998 to study the Martian climate, Martian atmosphere, and surface
changes and to act as the communications relay in the Mars Surveyor '98
program for Mars Polar Lander. However, on September 23, 1999,
communication with the spacecraft was lost as the spacecraft went into
orbital insertion, due to ground-based computer software which produced
output in non-SI units of pound (force)-seconds (lbf·s) instead of the
SI units of newton-seconds (N·s) specified in the contract between NASA
and Lockheed. The spacecraft encountered Mars on a trajectory that
brought it too close to the planet, causing it to pass through the upper
atmosphere and disintegrate."
(https://en.wikipedia.org/wiki/Mars\_Climate\_Orbiter\#Cause\_of\_failure)

\emph{Therac 25}

"Between June 1985 and January 1987, a computer-controlled radiation
therapy machine, called the Therac-25, massively overdosed six people.
These accidents have been described as the worst in the 35-year history
of medical accelators.

...

On the Therac-25, the part of the computer program that is often
referred to as the "house-keeper task" continuously checked to see
whether the turntable was correctly positioned. A zero on the counter
indicated to the technician that the turntable was in the correct
position. Any value other than zero meant that it wasn't, and that
treatment couldn't begin. The computer would then make the necessary
corrections and the counter would reset itself to zero.

But the highest value the counter could register was 255. If the program
reached 256 checks, the counter automatically clicked back to zero, the
same way that a car odometer turns over to zero after you've driven more
than 99,999.99 kilometres. For that split second, the Therac-25 believed
it was safe to proceed when, in fact, it wasn't. If the technician hit
the "set" button to begin treatment at that precise moment, the
turntable would be in the wrong position and the patient would be struck
by a raw beam."

    \subsubsection{Testing and Software Quality
Assurance}\label{testing-and-software-quality-assurance}

\paragraph{The Concept}\label{the-concept}

"Software quality assurance (SQA) consists of a means of monitoring the
software engineering processes and methods used to ensure quality. The
methods by which this is accomplished are many and varied, and may
include ensuring conformance to one or more standards, such as ISO 9000
or a model such as CMMI.

SQA encompasses the entire software development process, which includes
processes such as requirements definition, software design, coding,
source code control, code reviews, software configuration management,
testing, release management, and product integration. SQA is organized
into goals, commitments, abilities, activities, measurements, and
verifications."
(https://en.wikipedia.org/wiki/Software\_quality\_assurance)

\begin{itemize}
\item
  This is a massively complex topic in Computer Science.
\item
  We do not have time to cover in any detail.
\item
  We will, however, think about the concepts and some general
  guidelines.
\item
  My predominant rule of thumb is \textbf{follow} {[}\textbf{Gall's
  Law}{]}(https://en.wikipedia.org/wiki/John\_Gall\_(author)
\end{itemize}

"A complex system that works is invariably found to have evolved from a
simple system that worked. A complex system designed from scratch never
works and cannot be patched up to make it work. You have to start over
with a working simple system. -- John Gall (1975)

    \paragraph{An Example}\label{an-example}

Remember the 0-1 Knapsack Problem? The exact solution code is ...

    \begin{Verbatim}[commandchars=\\\{\}]
{\color{incolor}In [{\color{incolor} }]:} \PY{k+kn}{from} \PY{n+nn}{itertools} \PY{k}{import} \PY{n}{combinations}
        \PY{k+kn}{import} \PY{n+nn}{time}
        
        \PY{k}{def} \PY{n+nf}{anycomb}\PY{p}{(}\PY{n}{items}\PY{p}{)}\PY{p}{:}
            \PY{l+s+s1}{\PYZsq{}}\PY{l+s+s1}{ return combinations of any length from the items }\PY{l+s+s1}{\PYZsq{}}
            \PY{k}{return} \PY{p}{(} \PY{n}{comb}
                     \PY{k}{for} \PY{n}{r} \PY{o+ow}{in} \PY{n+nb}{range}\PY{p}{(}\PY{l+m+mi}{1}\PY{p}{,} \PY{n+nb}{len}\PY{p}{(}\PY{n}{items}\PY{p}{)}\PY{o}{+}\PY{l+m+mi}{1}\PY{p}{)}
                     \PY{k}{for} \PY{n}{comb} \PY{o+ow}{in} \PY{n}{combinations}\PY{p}{(}\PY{n}{items}\PY{p}{,} \PY{n}{r}\PY{p}{)}
                     \PY{p}{)}
         
        \PY{k}{def} \PY{n+nf}{totalvalue}\PY{p}{(}\PY{n}{comb}\PY{p}{)}\PY{p}{:}
            \PY{l+s+s1}{\PYZsq{}}\PY{l+s+s1}{ Totalise a particular combination of items}\PY{l+s+s1}{\PYZsq{}}
            \PY{n}{totwt} \PY{o}{=} \PY{n}{totval} \PY{o}{=} \PY{l+m+mi}{0}
            \PY{k}{for} \PY{n}{item}\PY{p}{,} \PY{n}{wt}\PY{p}{,} \PY{n}{val} \PY{o+ow}{in} \PY{n}{comb}\PY{p}{:}
                \PY{n}{totwt}  \PY{o}{+}\PY{o}{=} \PY{n}{wt}
                \PY{n}{totval} \PY{o}{+}\PY{o}{=} \PY{n}{val}
            \PY{k}{return} \PY{p}{(}\PY{n}{totval}\PY{p}{,} \PY{o}{\PYZhy{}}\PY{n}{totwt}\PY{p}{)} \PY{k}{if} \PY{n}{totwt} \PY{o}{\PYZlt{}}\PY{o}{=} \PY{l+m+mi}{500} \PY{k}{else} \PY{p}{(}\PY{l+m+mi}{0}\PY{p}{,} \PY{l+m+mi}{0}\PY{p}{)}
        
        \PY{c+c1}{\PYZsh{} Program/algorithm example input data. Real solution would get}
        \PY{c+c1}{\PYZsh{} data from user input, file, etc.}
        \PY{n}{items} \PY{o}{=} \PY{p}{(}
            \PY{p}{(}\PY{l+s+s2}{\PYZdq{}}\PY{l+s+s2}{map}\PY{l+s+s2}{\PYZdq{}}\PY{p}{,} \PY{l+m+mi}{9}\PY{p}{,} \PY{l+m+mi}{150}\PY{p}{)}\PY{p}{,} \PY{p}{(}\PY{l+s+s2}{\PYZdq{}}\PY{l+s+s2}{compass}\PY{l+s+s2}{\PYZdq{}}\PY{p}{,} \PY{l+m+mi}{13}\PY{p}{,} \PY{l+m+mi}{35}\PY{p}{)}\PY{p}{,} \PY{p}{(}\PY{l+s+s2}{\PYZdq{}}\PY{l+s+s2}{water}\PY{l+s+s2}{\PYZdq{}}\PY{p}{,} \PY{l+m+mi}{153}\PY{p}{,} \PY{l+m+mi}{200}\PY{p}{)}\PY{p}{,} \PY{p}{(}\PY{l+s+s2}{\PYZdq{}}\PY{l+s+s2}{sandwich}\PY{l+s+s2}{\PYZdq{}}\PY{p}{,} \PY{l+m+mi}{50}\PY{p}{,} \PY{l+m+mi}{160}\PY{p}{)}\PY{p}{,}
            \PY{p}{(}\PY{l+s+s2}{\PYZdq{}}\PY{l+s+s2}{glucose}\PY{l+s+s2}{\PYZdq{}}\PY{p}{,} \PY{l+m+mi}{15}\PY{p}{,} \PY{l+m+mi}{60}\PY{p}{)}\PY{p}{,} \PY{p}{(}\PY{l+s+s2}{\PYZdq{}}\PY{l+s+s2}{tin}\PY{l+s+s2}{\PYZdq{}}\PY{p}{,} \PY{l+m+mi}{68}\PY{p}{,} \PY{l+m+mi}{45}\PY{p}{)}\PY{p}{,} \PY{p}{(}\PY{l+s+s2}{\PYZdq{}}\PY{l+s+s2}{banana}\PY{l+s+s2}{\PYZdq{}}\PY{p}{,} \PY{l+m+mi}{27}\PY{p}{,} \PY{l+m+mi}{60}\PY{p}{)}\PY{p}{,} \PY{p}{(}\PY{l+s+s2}{\PYZdq{}}\PY{l+s+s2}{apple}\PY{l+s+s2}{\PYZdq{}}\PY{p}{,} \PY{l+m+mi}{39}\PY{p}{,} \PY{l+m+mi}{40}\PY{p}{)}\PY{p}{,}
           \PY{p}{(}\PY{l+s+s2}{\PYZdq{}}\PY{l+s+s2}{cheese}\PY{l+s+s2}{\PYZdq{}}\PY{p}{,} \PY{l+m+mi}{23}\PY{p}{,} \PY{l+m+mi}{30}\PY{p}{)}\PY{p}{,} \PY{p}{(}\PY{l+s+s2}{\PYZdq{}}\PY{l+s+s2}{beer}\PY{l+s+s2}{\PYZdq{}}\PY{p}{,} \PY{l+m+mi}{52}\PY{p}{,} \PY{l+m+mi}{10}\PY{p}{)}\PY{p}{,} \PY{p}{(}\PY{l+s+s2}{\PYZdq{}}\PY{l+s+s2}{suntan cream}\PY{l+s+s2}{\PYZdq{}}\PY{p}{,} \PY{l+m+mi}{11}\PY{p}{,} \PY{l+m+mi}{70}\PY{p}{)}\PY{p}{,} \PY{p}{(}\PY{l+s+s2}{\PYZdq{}}\PY{l+s+s2}{camera}\PY{l+s+s2}{\PYZdq{}}\PY{p}{,} \PY{l+m+mi}{32}\PY{p}{,} \PY{l+m+mi}{30}\PY{p}{)}\PY{p}{,}
            \PY{p}{(}\PY{l+s+s2}{\PYZdq{}}\PY{l+s+s2}{t\PYZhy{}shirt}\PY{l+s+s2}{\PYZdq{}}\PY{p}{,} \PY{l+m+mi}{24}\PY{p}{,} \PY{l+m+mi}{15}\PY{p}{)}\PY{p}{,} \PY{p}{(}\PY{l+s+s2}{\PYZdq{}}\PY{l+s+s2}{trousers}\PY{l+s+s2}{\PYZdq{}}\PY{p}{,} \PY{l+m+mi}{48}\PY{p}{,} \PY{l+m+mi}{10}\PY{p}{)}\PY{p}{,} \PY{p}{(}\PY{l+s+s2}{\PYZdq{}}\PY{l+s+s2}{umbrella}\PY{l+s+s2}{\PYZdq{}}\PY{p}{,} \PY{l+m+mi}{73}\PY{p}{,} \PY{l+m+mi}{40}\PY{p}{)}\PY{p}{,}
            \PY{p}{(}\PY{l+s+s2}{\PYZdq{}}\PY{l+s+s2}{waterproof trousers}\PY{l+s+s2}{\PYZdq{}}\PY{p}{,} \PY{l+m+mi}{42}\PY{p}{,} \PY{l+m+mi}{70}\PY{p}{)}\PY{p}{,} \PY{p}{(}\PY{l+s+s2}{\PYZdq{}}\PY{l+s+s2}{waterproof overclothes}\PY{l+s+s2}{\PYZdq{}}\PY{p}{,} \PY{l+m+mi}{43}\PY{p}{,} \PY{l+m+mi}{75}\PY{p}{)}\PY{p}{,}
            \PY{p}{(}\PY{l+s+s2}{\PYZdq{}}\PY{l+s+s2}{note\PYZhy{}case}\PY{l+s+s2}{\PYZdq{}}\PY{p}{,} \PY{l+m+mi}{22}\PY{p}{,} \PY{l+m+mi}{80}\PY{p}{)}\PY{p}{,} \PY{p}{(}\PY{l+s+s2}{\PYZdq{}}\PY{l+s+s2}{sunglasses}\PY{l+s+s2}{\PYZdq{}}\PY{p}{,} \PY{l+m+mi}{7}\PY{p}{,} \PY{l+m+mi}{20}\PY{p}{)}\PY{p}{,} \PY{p}{(}\PY{l+s+s2}{\PYZdq{}}\PY{l+s+s2}{towel}\PY{l+s+s2}{\PYZdq{}}\PY{p}{,} \PY{l+m+mi}{18}\PY{p}{,} \PY{l+m+mi}{12}\PY{p}{)}\PY{p}{,}
            \PY{p}{(}\PY{l+s+s2}{\PYZdq{}}\PY{l+s+s2}{socks}\PY{l+s+s2}{\PYZdq{}}\PY{p}{,} \PY{l+m+mi}{4}\PY{p}{,} \PY{l+m+mi}{50}\PY{p}{)}\PY{p}{,} \PY{p}{(}\PY{l+s+s2}{\PYZdq{}}\PY{l+s+s2}{book}\PY{l+s+s2}{\PYZdq{}}\PY{p}{,} \PY{l+m+mi}{30}\PY{p}{,} \PY{l+m+mi}{10}\PY{p}{)}\PY{p}{,} \PY{p}{(}\PY{l+s+s2}{\PYZdq{}}\PY{l+s+s2}{tent}\PY{l+s+s2}{\PYZdq{}}\PY{p}{,} \PY{l+m+mi}{50}\PY{p}{,} \PY{l+m+mi}{50}\PY{p}{)}\PY{p}{,} \PY{p}{(}\PY{l+s+s2}{\PYZdq{}}\PY{l+s+s2}{matches}\PY{l+s+s2}{\PYZdq{}}\PY{p}{,} \PY{l+m+mi}{5}\PY{p}{,}\PY{l+m+mi}{20}\PY{p}{)}\PY{p}{,}
            \PY{p}{(}\PY{l+s+s2}{\PYZdq{}}\PY{l+s+s2}{boots}\PY{l+s+s2}{\PYZdq{}}\PY{p}{,} \PY{l+m+mi}{30}\PY{p}{,} \PY{l+m+mi}{30}\PY{p}{)}\PY{p}{,} \PY{p}{(}\PY{l+s+s2}{\PYZdq{}}\PY{l+s+s2}{flare}\PY{l+s+s2}{\PYZdq{}}\PY{p}{,} \PY{l+m+mi}{10}\PY{p}{,} \PY{l+m+mi}{25}\PY{p}{)}\PY{p}{,} \PY{p}{(}\PY{l+s+s2}{\PYZdq{}}\PY{l+s+s2}{mirror}\PY{l+s+s2}{\PYZdq{}}\PY{p}{,} \PY{l+m+mi}{50}\PY{p}{,} \PY{l+m+mi}{50}\PY{p}{)}
            \PY{p}{)}
        
        \PY{n}{start} \PY{o}{=} \PY{n}{time}\PY{o}{.}\PY{n}{time}\PY{p}{(}\PY{p}{)}
        \PY{n+nb}{print} \PY{p}{(}\PY{l+s+s2}{\PYZdq{}}\PY{l+s+s2}{Time = }\PY{l+s+s2}{\PYZdq{}}\PY{p}{,} \PY{n}{start}\PY{p}{)}    
        \PY{n}{bagged} \PY{o}{=} \PY{n+nb}{max}\PY{p}{(} \PY{n}{anycomb}\PY{p}{(}\PY{n}{items}\PY{p}{)}\PY{p}{,} \PY{n}{key}\PY{o}{=}\PY{n}{totalvalue}\PY{p}{)} \PY{c+c1}{\PYZsh{} max val or min wt if values equal}
        \PY{n}{done} \PY{o}{=} \PY{n}{time}\PY{o}{.}\PY{n}{time}\PY{p}{(}\PY{p}{)}
        \PY{n}{elapsed}\PY{o}{=}\PY{n}{done}\PY{o}{\PYZhy{}}\PY{n}{start}
        \PY{n+nb}{print}\PY{p}{(}\PY{l+s+s2}{\PYZdq{}}\PY{l+s+s2}{Done time =}\PY{l+s+s2}{\PYZdq{}}\PY{p}{,} \PY{n}{done}\PY{p}{,} \PY{l+s+s2}{\PYZdq{}}\PY{l+s+s2}{ elapsed = }\PY{l+s+s2}{\PYZdq{}}\PY{p}{,} \PY{n}{elapsed} \PY{p}{)}
        \PY{n+nb}{print}\PY{p}{(}\PY{l+s+s2}{\PYZdq{}}\PY{l+s+s2}{Bagged the following items}\PY{l+s+se}{\PYZbs{}n}\PY{l+s+s2}{  }\PY{l+s+s2}{\PYZdq{}} \PY{o}{+}
              \PY{l+s+s1}{\PYZsq{}}\PY{l+s+se}{\PYZbs{}n}\PY{l+s+s1}{  }\PY{l+s+s1}{\PYZsq{}}\PY{o}{.}\PY{n}{join}\PY{p}{(}\PY{n+nb}{sorted}\PY{p}{(}\PY{n}{item} \PY{k}{for} \PY{n}{item}\PY{p}{,}\PY{n}{\PYZus{}}\PY{p}{,}\PY{n}{\PYZus{}} \PY{o+ow}{in} \PY{n}{bagged}\PY{p}{)}\PY{p}{)}\PY{p}{)}
        \PY{n}{val}\PY{p}{,} \PY{n}{wt} \PY{o}{=} \PY{n}{totalvalue}\PY{p}{(}\PY{n}{bagged}\PY{p}{)}
        \PY{n+nb}{print}\PY{p}{(}\PY{l+s+s2}{\PYZdq{}}\PY{l+s+s2}{for a total value of }\PY{l+s+si}{\PYZpc{}i}\PY{l+s+s2}{ and a total weight of }\PY{l+s+si}{\PYZpc{}i}\PY{l+s+s2}{\PYZdq{}} \PY{o}{\PYZpc{}} \PY{p}{(}\PY{n}{val}\PY{p}{,} \PY{o}{\PYZhy{}}\PY{n}{wt}\PY{p}{)}\PY{p}{)}
\end{Verbatim}


    \begin{itemize}
\item
  I would not write this program all at once.
\item
  I would start by writing two smaller programs to
  \href{https://en.wikipedia.org/wiki/Unit_testing}{unit test} the
  smaller functional units:

  \begin{itemize}
  \tightlist
  \item
    anycomb function.
  \item
    totalvalue function.
  \end{itemize}
\end{itemize}

\emph{Unit test anycomb:}

    \begin{Verbatim}[commandchars=\\\{\}]
{\color{incolor}In [{\color{incolor} }]:} \PY{k+kn}{from} \PY{n+nn}{itertools} \PY{k}{import} \PY{n}{combinations}
        \PY{k+kn}{import} \PY{n+nn}{time}
        
        \PY{k}{def} \PY{n+nf}{anycomb}\PY{p}{(}\PY{n}{items}\PY{p}{)}\PY{p}{:}
            \PY{l+s+s1}{\PYZsq{}}\PY{l+s+s1}{ return combinations of any length from the items }\PY{l+s+s1}{\PYZsq{}}
            \PY{k}{return} \PY{p}{(} \PY{n}{comb}
                     \PY{k}{for} \PY{n}{r} \PY{o+ow}{in} \PY{n+nb}{range}\PY{p}{(}\PY{l+m+mi}{1}\PY{p}{,} \PY{n+nb}{len}\PY{p}{(}\PY{n}{items}\PY{p}{)}\PY{o}{+}\PY{l+m+mi}{1}\PY{p}{)}
                     \PY{k}{for} \PY{n}{comb} \PY{o+ow}{in} \PY{n}{combinations}\PY{p}{(}\PY{n}{items}\PY{p}{,} \PY{n}{r}\PY{p}{)}
                     \PY{p}{)}
        
        \PY{n}{test\PYZus{}items} \PY{o}{=} \PY{p}{(}
            \PY{p}{(}\PY{l+s+s2}{\PYZdq{}}\PY{l+s+s2}{map}\PY{l+s+s2}{\PYZdq{}}\PY{p}{,} \PY{l+m+mi}{9}\PY{p}{,} \PY{l+m+mi}{150}\PY{p}{)}\PY{p}{,} \PY{p}{(}\PY{l+s+s2}{\PYZdq{}}\PY{l+s+s2}{compass}\PY{l+s+s2}{\PYZdq{}}\PY{p}{,} \PY{l+m+mi}{13}\PY{p}{,} \PY{l+m+mi}{35}\PY{p}{)}\PY{p}{,} \PY{p}{(}\PY{l+s+s2}{\PYZdq{}}\PY{l+s+s2}{water}\PY{l+s+s2}{\PYZdq{}}\PY{p}{,} \PY{l+m+mi}{153}\PY{p}{,} \PY{l+m+mi}{200}\PY{p}{)}\PY{p}{)}
        
        \PY{n}{result} \PY{o}{=} \PY{n}{anycomb}\PY{p}{(}\PY{n}{test\PYZus{}items}\PY{p}{)}
        
        \PY{n+nb}{print}\PY{p}{(}\PY{l+s+s2}{\PYZdq{}}\PY{l+s+s2}{Result = }\PY{l+s+s2}{\PYZdq{}}\PY{p}{)}
        \PY{n+nb}{print}\PY{p}{(}\PY{o}{*}\PY{n}{result}\PY{p}{,} \PY{n}{sep}\PY{o}{=}\PY{l+s+s1}{\PYZsq{}}\PY{l+s+se}{\PYZbs{}n}\PY{l+s+s1}{\PYZsq{}}\PY{p}{)}
\end{Verbatim}


    This is correct. I can manually verify for small input sets.

Obviously, I would test more than once and with tricky combinations.

    \emph{Unit test totalvalue:} BTW, not sure why the developer chose to
return -total\_weight.

    \begin{Verbatim}[commandchars=\\\{\}]
{\color{incolor}In [{\color{incolor} }]:} \PY{k}{def} \PY{n+nf}{totalvalue}\PY{p}{(}\PY{n}{comb}\PY{p}{)}\PY{p}{:}
            \PY{l+s+s1}{\PYZsq{}}\PY{l+s+s1}{ Totalise a particular combination of items}\PY{l+s+s1}{\PYZsq{}}
            \PY{n}{totwt} \PY{o}{=} \PY{n}{totval} \PY{o}{=} \PY{l+m+mi}{0}
            \PY{k}{for} \PY{n}{item}\PY{p}{,} \PY{n}{wt}\PY{p}{,} \PY{n}{val} \PY{o+ow}{in} \PY{n}{comb}\PY{p}{:}
                \PY{n}{totwt}  \PY{o}{+}\PY{o}{=} \PY{n}{wt}
                \PY{n}{totval} \PY{o}{+}\PY{o}{=} \PY{n}{val}
            \PY{k}{return} \PY{p}{(}\PY{n}{totval}\PY{p}{,} \PY{o}{\PYZhy{}}\PY{n}{totwt}\PY{p}{)} \PY{k}{if} \PY{n}{totwt} \PY{o}{\PYZlt{}}\PY{o}{=} \PY{l+m+mi}{500} \PY{k}{else} \PY{p}{(}\PY{l+m+mi}{0}\PY{p}{,} \PY{l+m+mi}{0}\PY{p}{)}
        
        \PY{n}{test1} \PY{o}{=} \PY{p}{(}\PY{p}{(}\PY{l+s+s1}{\PYZsq{}}\PY{l+s+s1}{map}\PY{l+s+s1}{\PYZsq{}}\PY{p}{,} \PY{l+m+mi}{9}\PY{p}{,} \PY{l+m+mi}{150}\PY{p}{)}\PY{p}{,} \PY{p}{(}\PY{l+s+s1}{\PYZsq{}}\PY{l+s+s1}{water}\PY{l+s+s1}{\PYZsq{}}\PY{p}{,} \PY{l+m+mi}{153}\PY{p}{,} \PY{l+m+mi}{200}\PY{p}{)}\PY{p}{)}
        
        \PY{n+nb}{print}\PY{p}{(}\PY{l+s+s2}{\PYZdq{}}\PY{l+s+s2}{totalvalue test 1 = }\PY{l+s+s2}{\PYZdq{}}\PY{p}{,} \PY{n}{totalvalue}\PY{p}{(}\PY{n}{test1}\PY{p}{)}\PY{p}{)}
\end{Verbatim}


    Finally, make sure you code checks for erroneous input.

    \begin{Verbatim}[commandchars=\\\{\}]
{\color{incolor}In [{\color{incolor} }]:} \PY{n}{test2}\PY{o}{=}\PY{p}{(}\PY{p}{(}\PY{l+s+s1}{\PYZsq{}}\PY{l+s+s1}{map}\PY{l+s+s1}{\PYZsq{}}\PY{p}{,} \PY{o}{\PYZhy{}}\PY{l+m+mi}{900}\PY{p}{,} \PY{l+m+mi}{150}\PY{p}{)}\PY{p}{,} \PY{p}{(}\PY{l+s+s1}{\PYZsq{}}\PY{l+s+s1}{water}\PY{l+s+s1}{\PYZsq{}}\PY{p}{,} \PY{l+m+mi}{1100}\PY{p}{,} \PY{l+m+mi}{200}\PY{p}{)}\PY{p}{)}
        \PY{n+nb}{print}\PY{p}{(}\PY{l+s+s2}{\PYZdq{}}\PY{l+s+s2}{totalvalue test 2 = }\PY{l+s+s2}{\PYZdq{}}\PY{p}{,} \PY{n}{totalvalue}\PY{p}{(}\PY{n}{test2}\PY{p}{)}\PY{p}{)}
\end{Verbatim}


    This appears to be a reasonable answer, despite the fact that - The
maximum weight allowed is 500 - Water weighs 1100

Fools will inevitably enter bad data or do something wrong.

\emph{"It is impossible to make anything foolproof because fools are so
ingenious."} Corollary to Murphy's Law.

This may be true, but you need to try.

BTW, there is also Smith's Law. \textbf{"Murphy was an optimist."}


    % Add a bibliography block to the postdoc
    
    
    
    \end{document}
